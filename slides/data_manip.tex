% -*- mode: noweb; ess-noweb-default-code-mode: R-mode; -*-
\documentclass[paper=screen,mathserif]{beamer}\usepackage[]{graphicx}\usepackage[]{color}
%% maxwidth is the original width if it is less than linewidth
%% otherwise use linewidth (to make sure the graphics do not exceed the margin)
\makeatletter
\def\maxwidth{ %
  \ifdim\Gin@nat@width>\linewidth
    \linewidth
  \else
    \Gin@nat@width
  \fi
}
\makeatother

\definecolor{fgcolor}{rgb}{0.345, 0.345, 0.345}
\newcommand{\hlnum}[1]{\textcolor[rgb]{0.686,0.059,0.569}{#1}}%
\newcommand{\hlstr}[1]{\textcolor[rgb]{0.192,0.494,0.8}{#1}}%
\newcommand{\hlcom}[1]{\textcolor[rgb]{0.678,0.584,0.686}{\textit{#1}}}%
\newcommand{\hlopt}[1]{\textcolor[rgb]{0,0,0}{#1}}%
\newcommand{\hlstd}[1]{\textcolor[rgb]{0.345,0.345,0.345}{#1}}%
\newcommand{\hlkwa}[1]{\textcolor[rgb]{0.161,0.373,0.58}{\textbf{#1}}}%
\newcommand{\hlkwb}[1]{\textcolor[rgb]{0.69,0.353,0.396}{#1}}%
\newcommand{\hlkwc}[1]{\textcolor[rgb]{0.333,0.667,0.333}{#1}}%
\newcommand{\hlkwd}[1]{\textcolor[rgb]{0.737,0.353,0.396}{\textbf{#1}}}%

\usepackage{framed}
\makeatletter
\newenvironment{kframe}{%
 \def\at@end@of@kframe{}%
 \ifinner\ifhmode%
  \def\at@end@of@kframe{\end{minipage}}%
  \begin{minipage}{\columnwidth}%
 \fi\fi%
 \def\FrameCommand##1{\hskip\@totalleftmargin \hskip-\fboxsep
 \colorbox{shadecolor}{##1}\hskip-\fboxsep
     % There is no \\@totalrightmargin, so:
     \hskip-\linewidth \hskip-\@totalleftmargin \hskip\columnwidth}%
 \MakeFramed {\advance\hsize-\width
   \@totalleftmargin\z@ \linewidth\hsize
   \@setminipage}}%
 {\par\unskip\endMakeFramed%
 \at@end@of@kframe}
\makeatother

\definecolor{shadecolor}{rgb}{.97, .97, .97}
\definecolor{messagecolor}{rgb}{0, 0, 0}
\definecolor{warningcolor}{rgb}{1, 0, 1}
\definecolor{errorcolor}{rgb}{1, 0, 0}
\newenvironment{knitrout}{}{} % an empty environment to be redefined in TeX

\usepackage{alltt}

\usetheme{CambridgeUS} 
\useinnertheme{circles}
\useoutertheme[footline=authortitle,subsection = false]{miniframes}
\setbeamercolor{palette tertiary}{fg=white, bg=white!42!black}
\setbeamercolor{alerted text}{fg=red!73!black}

%%%%%%
%\usepackage{Sweave}
\usepackage{natbib}     % for references
\usepackage[osf]{sourcesanspro}
\usepackage{sourcecodepro}
\usepackage{booktabs}
\usepackage{eulervm}
%\renewcommand{\ttdefault}{sourcecodepro}
\usepackage{import}
\usepackage{prodint}
\usepackage{bbm}
\usepackage{tabularx}
\usepackage{dcolumn}
\usepackage{color}
\usepackage{booktabs}
\usepackage{graphicx,rotating,epsfig,multirow,multicol,hhline}
\usepackage{amsmath,amsthm,amssymb,amsfonts}

\newcommand{\subfloat}[2][need a sub-caption]{\subcaptionbox{#1}{#2}}

\usepackage{listings}
\lstset{
  basicstyle=\tiny\ttfamily, % Standardschrift
  % numbers=left,               % Ort der Zeilennummern
  %numberstyle=\tiny,          % Stil der Zeilennummern
  % stepnumber=2,               % Abstand zwischen den Zeilennummern
  numbersep=5pt,              % Abstand der Nummern zum Text
  tabsize=2,                  % Groesse von Tabs
  extendedchars=true,         %
  breaklines=true,            % Zeilen werden Umgebrochen
  keywordstyle=\color{blue},
  frame=b,         
  stringstyle=\color{white}\ttfamily, % Farbe der String
  showspaces=false,           % Leerzeichen anzeigen ?
  showtabs=false,             % Tabs anzeigen ?
}

\usepackage{subcaption}

\newcommand{\ft}[1]{\frametitle{#1}}
\newcommand{\fst}[1]{\framesubtitle{#1}}

\newenvironment{xframe}[1][]
{\begin{frame}[fragile,environment=xframe]
    \frametitle{#1}}
  {\end{frame}}

\title[Data Manipulation]{Data Manipulation}

\author{Arthur Allignol}

\institute[]{\scriptsize{\url{arthur.allignol@uni-ulm.de}}}

\date{}
%%%%%%

\makeatletter
\setbeamertemplate{footline}
{
  \leavevmode%
  \hbox{%
  \begin{beamercolorbox}[wd=.333333\paperwidth,ht=2.25ex,dp=1ex,center]{author in head/foot}%
    \usebeamerfont{author in head/foot}\insertshortauthor%~~\beamer@ifempty{\insertshortinstitute}{}{(\insertshortinstitute)}
  \end{beamercolorbox}%
  \begin{beamercolorbox}[wd=.333333\paperwidth,ht=2.25ex,dp=1ex,center]{title in head/foot}%
    \usebeamerfont{title in head/foot}\insertshorttitle
  \end{beamercolorbox}%
  \begin{beamercolorbox}[wd=.333333\paperwidth,ht=2.25ex,dp=1ex,right]{date in head/foot}%
    \usebeamerfont{date in head/foot}\insertshortdate{}\hspace*{2em}
    \insertframenumber\hspace*{2ex} 
  \end{beamercolorbox}}%
  \vskip0pt%
}
\makeatother


\AtBeginSection[]
{
  \begin{frame}
    \frametitle{Table of Contents}
    \tableofcontents[currentsection]
  \end{frame}
}
\IfFileExists{upquote.sty}{\usepackage{upquote}}{}
\begin{document}

%%%%%% title page
\newcommand{\titlep}{yes}  % for titlepagelayout

{
\renewcommand{\insertframenumber}{}   % no page number on titlepage
\begin{frame}
\addtocounter{framenumber}{-1}
\titlepage
\end{frame}
}



\section{Reading and Writing Data}

\begin{frame}[fragile]
  \ft{Reading Data} 
  \fst{{\tt scan()}}
    
  \verb=scan()= reads data into a vector or list from the console or
  from file
    
    \begin{itemize}
    \item \verb=scan= is more appropriate when all the data to be read
      are of the same mode
    \item Arguments:
      \begin{itemize}
      \item {\tt file}: Name of a file. When {\tt ""}, reads from the
        console
      \item {\tt what}: Type of {\tt what} gives the type of data to
        be read.
        
        {\tt what} can also be a list
      \item {\tt scan} calls can be embedded in a call to {\tt matrix}
\begin{knitrout}\scriptsize
\definecolor{shadecolor}{rgb}{0.969, 0.969, 0.969}\color{fgcolor}\begin{kframe}
\begin{alltt}
\hlkwd{matrix}\hlstd{(}\hlkwd{scan}\hlstd{(),} \hlkwc{ncol} \hlstd{=} \hlnum{3}\hlstd{,} \hlkwc{byrow} \hlstd{=} \hlnum{TRUE}\hlstd{)}
\end{alltt}
\end{kframe}
\end{knitrout}
\end{itemize}
\end{itemize}
\end{frame}


\begin{frame}[fragile]
  \ft{Data Frames}
  \fst{{\tt read.table}}
  \begin{itemize}
  \item The {\tt read.table} function is used to read data into R in
    the form of a data frame, i.e., data with mixed modes
  \item {\tt read.table} expects each field (variable) to be
    separated by separators (by default,  spaces, tabs, newlines or
    carriage returns)
    \begin{itemize}
    \item The {\tt sep} argument can be used to specify an
      alternative separator
    \end{itemize}
  \item R provides convenience functions for reading comma- and
    tab-separated data 
  \end{itemize}
  \begin{center}
    \begin{tabular}{p{4cm}p{6cm}}
      {\tt read.csv}   & Separated by {\tt ,} \\
      {\tt read.csv2}  & Separated by {\tt ;} decimal point {\tt ,} \\
      {\tt read.delim} & Separated by tabs    \\
      {\tt read.delim2} & Separated by tabs, decimal point {\tt ,}    \\
    \end{tabular}
  \end{center}
  {\small These functions are wrappers for {\tt read.table} with the
    {\tt sep} argument set appropriately}
\end{frame}

\begin{frame}[fragile]
  \ft{Data Frames}
  \fst{{\tt read.table}: Useful options}
  
  \begin{center} 
    {\scriptsize
    \begin{tabular}{lp{7cm}} 
      \toprule
      {\tt file}                 & File to be read or a {\em connection}      \\
      {\tt sep}                  & e.g., \verb="\t"=, {\tt ","}               \\
      {\tt dec}                  & Specify decimal point (default is {\tt .}) \\
      {\tt header}               & {\tt TRUE} if the the first line are the column
      names (default to {\tt TRUE} for {\tt read.csv}...)                     \\
      {\tt col.names}            & A vector of column names                   \\
      {\tt stringsAsFactors}     & Logical. If {\tt FALSE}, prevent the
      automatic conversion of character strings into factors                  \\
      {\tt na.strings}           & By default, {\tt NA}, {\tt NaN}, {\tt Inf}
      and {\tt -Inf} are considered as missing values. Change this
      behaviour using {\tt na.strings}                                        \\ 
      {\tt skip} and {\tt nrows} & Number of lines to skip and number
      of lines to read, respectively                                          \\
      {\tt fill}                 & If {\tt TRUE}, observations with
      fewer variables are filled with {\tt NA}s or blanks                     \\
      {\tt colClasses}           & Specify the modes of the columns to
      be read                                                                 \\
      \verb=fileEncoding=        & Encoding of the file. Useful for
      non ASCII characters from other platforms                         \\
      \bottomrule
    \end{tabular}}
  \end{center}
\end{frame}

\begin{frame}[fragile]
  \ft{Data Frames}
  \fst{Fixed Width Input Files}
  \begin{itemize}
  \item Files without delimiters but for which each variable is stored
    in one column
  \item Can be read in R using the {\tt read.fwf}
    \begin{itemize}
    \item {\tt file}: the file to be read
    \item {\tt widths}: vector containing the widths of the fields to
      be read
    \end{itemize}
  \end{itemize}
\end{frame}

\begin{frame}[fragile]
  \ft{{\tt readLines}}
  \begin{itemize}
  \item {\tt readLines} reads some or all text lines from a file or a
    connection
  \item Useful, for example, if only some lines of an enormous file
    need to be read
    \begin{itemize}
    \item {\tt con}: a connection or file name
    \item {\tt n}: The maximal number of lines to read. Negative
      values indicate that the whole file should be read
    \end{itemize}
  \end{itemize}
\end{frame}

\begin{frame}[fragile]
  \ft{Connections}
  \begin{itemize}
  \item Connections provide a flexible way to read data from a variety
    of sources
  \item Connections can be used as input for, e.g., {\tt read.table}
    or {\tt readLines}
  \end{itemize}
  \begin{center}
  \begin{tabular}{lp{7cm}}
    \toprule
    Function & Data source \\
    \midrule
    {\tt file} & Local files \\
    {\tt url} & Remote read via http ot ftp \\
    {\tt gzfile} & Local gzipped file \\
    {\tt unz} & Local zip archive \\
    {\tt pipe} & Output from a command \\
    \dots & \\
    \bottomrule
  \end{tabular}
\end{center}  
\end{frame}

\begin{frame}[fragile]
  \ft{Connections}

\begin{knitrout}\scriptsize
\definecolor{shadecolor}{rgb}{0.969, 0.969, 0.969}\color{fgcolor}\begin{kframe}
\begin{alltt}
\hlstd{myCon} \hlkwb{<-} \hlkwd{url}\hlstd{(}\hlstr{"http://taz.de/"}\hlstd{)}
\hlstd{aa} \hlkwb{<-} \hlkwd{readLines}\hlstd{(myCon,} \hlnum{3}\hlstd{)}
\hlstd{aa}
\end{alltt}
\begin{verbatim}
## [1] "<!DOCTYPE html SYSTEM \"about:legacy-compat\">"                                                                                                             
## [2] "<html xmlns=\"http://www.w3.org/1999/xhtml\" xmlns:my=\"mynames\" xmlns:gf=\"http://www.taz.de/namespace/gitfilter\"><!-- DEBUG start 17:00:48+02:00--><!--"
## [3] "\t\tContent Management: openNewspaper www.opennewspaper.org based on TYPO3 www.typo3.org"
\end{verbatim}
\begin{alltt}
\hlkwd{close}\hlstd{(myCon)}
\end{alltt}
\end{kframe}
\end{knitrout}
\end{frame}

\begin{frame}[fragile]
  \ft{Other useful functions}
  \begin{center}
    \begin{tabular}{lp{6cm}}
      \toprule
      Function & Format \\
      \midrule
      \verb=load= & Read R data format {\tt .rda}, {\tt .RData} \\
      \verb=read.dta= & Read data saved by Stata ({\bf foreign} package)\\
      \verb=read.spss= & Read data from SPSS ({\bf foreign})\\
      \verb=read.ssd=, \verb=read.xport= & Read SAS files \\
      \bottomrule
    \end{tabular}
  \end{center}
  \begin{itemize}
  \item Read a data from Excel
    \begin{itemize}
    \item \alert{Export from Excel into a .csv file}
    \item On Windows, the {\bf RODBC} package permits to access Excel files
    \item {\bf gdata} package --- for all platforms. Requires some specific
      perl modules to be installed
    \item {\bf xlsx} package. Requires Java / installation tricky
    \item {\bf readxl} package. Not tested (released 15.4.2015)
    \end{itemize}
  \end{itemize}
\end{frame}

\begin{frame}[fragile]
  \ft{The {\bf readr} package}
  
  The {\bf readr} package provides alternatives to the base {\tt
    read.***} functions that are
  \begin{itemize}
  \item 10 times faster (according to
    \url{http://blog.rstudio.org/2015/04/09/readr-0-1-0/})
  \item more consistent
  \item more flexible column specification
  \item \dots
  \end{itemize}
  \begin{center}
    \begin{tabular}{p{4cm}p{6cm}}
      {\tt read\_csv}   & Separated by {\tt ,} \\
      {\tt read\_csv2}  & Separated by {\tt ;} decimal point {\tt ,} \\
      {\tt read\_tsv} & Separated by tabs    \\
      {\tt read\_delim} & Separated by arbitrary delimiter    \\
    \end{tabular}
  \end{center}
\end{frame}

\begin{frame}[fragile]
  \ft{The {\bf readr} package}
  
  Three important arguments (similar for all functions)
  \begin{description}
  \item[{\tt file}] File name
  \item[{\tt col\_names}] column name; equivalent to {\tt header}.
    \begin{itemize}
    \item {\tt TRUE} $\Leftrightarrow$ \verb|header = TRUE| in base R
    \item {\tt FALSE}
    \item A character vector to use as column names
    \end{itemize}
  \item[{\tt col\_types}] Override the default column types
  \end{description}\vspace{0.6cm}
  Default to \verb|stringsAsFactors = FALSE|!
\end{frame}

\begin{frame}[fragile]
  \ft{Writing Data}
  \begin{itemize}\setlength\parskip{10pt}
  \item The {\tt save} function can be used to save R objects
\begin{knitrout}\scriptsize
\definecolor{shadecolor}{rgb}{0.969, 0.969, 0.969}\color{fgcolor}\begin{kframe}
\begin{alltt}
\hlkwd{save}\hlstd{(a_data_set, x, y,} \hlkwc{file} \hlstd{=} \hlstr{"my_data.rda"}\hlstd{)}
\end{alltt}
\end{kframe}
\end{knitrout}
  \item The {\tt write} function
    \begin{itemize}
    \item Takes an R object and the name of a file or a connection as arguments
    \item Writes a ASCII representation of the object
    \item The {\tt ncolumns} argument specifies the number of values
      to write on each line
    \end{itemize}
  \end{itemize}
\end{frame}

\begin{frame}[fragile]
  \ft{Writing Data}
  \begin{itemize}
  \item The {\tt write.table} function
    \begin{itemize}
    \item Requires the name of a data set or matrix (if no file is
      specified, {\tt write.table()} writes into the console)
    \item The {\tt file} argument specifies the destination. {\tt
        file} can also be a connection
    \item {\tt row.names} and {\tt col.names} specify whether to write
      the rows' and columns' names, respectively
    \item {\tt sep} specifies the separator. Default is blank 
    \item {\tt write.csv} and {\tt write.csv2} available for writing
      comma-separated files
    \end{itemize}
  \end{itemize}
\end{frame}

\begin{frame}[fragile]
  \ft{Your Turn}
  \begin{itemize}
  \item The zip file {\tt data.zip} contains several versions of the
    same (fake) data set
    \begin{itemize}
    \item {\tt data1.rda} can be opened using \verb=load()= and will
      serve as reference
    \item {\tt data2} to {\tt data5} are ASCII files
    \item {\tt data6.xls} is an Excel file that is rather realistic
    \end{itemize}
  \item Read all the data sets in R and check whether they compare to
    the original using, e.g., 
\begin{knitrout}\scriptsize
\definecolor{shadecolor}{rgb}{0.969, 0.969, 0.969}\color{fgcolor}\begin{kframe}
\begin{alltt}
\hlkwd{all}\hlstd{(dd1} \hlopt{==} \hlstd{dd2)} \hlcom{# dd2 being another data set}
\end{alltt}
\end{kframe}
\end{knitrout}
 
  \end{itemize}
  
\end{frame}



\section{Factors and Dates}

\begin{frame}[fragile]
  \ft{Factors}
  \begin{itemize}
  \item Factor are variables in R that take on a limited number of
    different values
    \begin{itemize}
    \item Categorical variables
    \item Ordinal variables
    \end{itemize}
  \item Factors are useful for statistical modelling as ordinal
    variables should be treated differently than continuous variables
  \item Factors are also useful for statistical report
    generation. Think SAS labels
  \end{itemize}
\end{frame}

\begin{frame}[fragile]
  \ft{Factors}
  \begin{itemize}
  \item Factors are stored internally as numeric values

  \item A corresponding set of characters is used for displaying
  \end{itemize}
\begin{knitrout}\scriptsize
\definecolor{shadecolor}{rgb}{0.969, 0.969, 0.969}\color{fgcolor}\begin{kframe}
\begin{alltt}
\hlstd{aa} \hlkwb{<-} \hlkwd{factor}\hlstd{(}\hlkwd{c}\hlstd{(}\hlstr{"cats"}\hlstd{,} \hlstr{"dogs"}\hlstd{,} \hlstr{"apples"}\hlstd{))}
\hlstd{aa}
\end{alltt}
\begin{verbatim}
## [1] cats   dogs   apples
## Levels: apples cats dogs
\end{verbatim}
\begin{alltt}
\hlkwd{as.integer}\hlstd{(aa)}
\end{alltt}
\begin{verbatim}
## [1] 2 3 1
\end{verbatim}
\end{kframe}
\end{knitrout}
\end{frame}

\begin{frame}[fragile]
  \ft{Factor Creation}
  \begin{itemize}
  \item Factors are created using the {\tt factor} function
  \item The {\tt levels} argument permits to control the order
  \item The {\tt labels} argument is used to change the levels' names
  \item \verb|ordered = TRUE| creates an ordered factor (ordinal
    variable)
  \end{itemize}
\begin{knitrout}\scriptsize
\definecolor{shadecolor}{rgb}{0.969, 0.969, 0.969}\color{fgcolor}\begin{kframe}
\begin{alltt}
\hlkwd{set.seed}\hlstd{(}\hlnum{21324}\hlstd{)}
\hlstd{data} \hlkwb{<-} \hlkwd{sample}\hlstd{(}\hlkwd{c}\hlstd{(}\hlnum{1}\hlstd{,} \hlnum{2}\hlstd{,} \hlnum{3}\hlstd{),} \hlnum{10}\hlstd{,} \hlnum{TRUE}\hlstd{)}
\hlstd{f0} \hlkwb{<-} \hlkwd{factor}\hlstd{(data)}
\hlstd{f1} \hlkwb{<-} \hlkwd{factor}\hlstd{(data,} \hlkwc{levels} \hlstd{=} \hlkwd{c}\hlstd{(}\hlnum{2}\hlstd{,} \hlnum{3}\hlstd{,} \hlnum{1}\hlstd{))}
\hlstd{f2} \hlkwb{<-} \hlkwd{factor}\hlstd{(data,} \hlkwc{labels} \hlstd{=} \hlkwd{c}\hlstd{(}\hlstr{"I"}\hlstd{,} \hlstr{"II"}\hlstd{,} \hlstr{"III"}\hlstd{))}
\hlstd{f3} \hlkwb{<-} \hlkwd{factor}\hlstd{(data,} \hlkwc{levels} \hlstd{=} \hlkwd{c}\hlstd{(}\hlnum{2}\hlstd{,} \hlnum{3}\hlstd{,} \hlnum{1}\hlstd{),}
             \hlkwc{labels} \hlstd{=} \hlkwd{c}\hlstd{(}\hlstr{"II"}\hlstd{,} \hlstr{"III"}\hlstd{,} \hlstr{"I"}\hlstd{))}
\end{alltt}
\end{kframe}
\end{knitrout}
\end{frame}

\begin{frame}[fragile]
\begin{knitrout}\scriptsize
\definecolor{shadecolor}{rgb}{0.969, 0.969, 0.969}\color{fgcolor}\begin{kframe}
\begin{alltt}
\hlkwd{table}\hlstd{(f0)}
\end{alltt}
\begin{verbatim}
## f0
## 1 2 3 
## 4 3 3
\end{verbatim}
\end{kframe}
\end{knitrout}
\begin{knitrout}\scriptsize
\definecolor{shadecolor}{rgb}{0.969, 0.969, 0.969}\color{fgcolor}\begin{kframe}
\begin{alltt}
\hlkwd{table}\hlstd{(f1)}
\end{alltt}
\begin{verbatim}
## f1
## 2 3 1 
## 3 3 4
\end{verbatim}
\end{kframe}
\end{knitrout}
\begin{knitrout}\scriptsize
\definecolor{shadecolor}{rgb}{0.969, 0.969, 0.969}\color{fgcolor}\begin{kframe}
\begin{alltt}
\hlkwd{table}\hlstd{(f2)}
\end{alltt}
\begin{verbatim}
## f2
##   I  II III 
##   4   3   3
\end{verbatim}
\end{kframe}
\end{knitrout}
\begin{knitrout}\scriptsize
\definecolor{shadecolor}{rgb}{0.969, 0.969, 0.969}\color{fgcolor}\begin{kframe}
\begin{alltt}
\hlkwd{table}\hlstd{(f3)}
\end{alltt}
\begin{verbatim}
## f3
##  II III   I 
##   3   3   4
\end{verbatim}
\end{kframe}
\end{knitrout}

\end{frame}

\begin{frame}[fragile]
  \ft{Factors}
  
  \begin{itemize}
  \item The \verb=levels()= function can be used to change the labels
    once a factor has been created
\begin{knitrout}\scriptsize
\definecolor{shadecolor}{rgb}{0.969, 0.969, 0.969}\color{fgcolor}\begin{kframe}
\begin{alltt}
\hlkwd{levels}\hlstd{(f0)} \hlkwb{<-} \hlkwd{c}\hlstd{(}\hlstr{"I"}\hlstd{,} \hlstr{"II"}\hlstd{,} \hlstr{"III"}\hlstd{)}
\hlstd{f0}
\end{alltt}
\begin{verbatim}
##  [1] II  I   III III III I   I   II  I   II 
## Levels: I II III
\end{verbatim}
\end{kframe}
\end{knitrout}
\item The reference level of a factor can be changed using the
  \verb=relevel= function
\begin{knitrout}\scriptsize
\definecolor{shadecolor}{rgb}{0.969, 0.969, 0.969}\color{fgcolor}\begin{kframe}
\begin{alltt}
\hlstd{f0} \hlkwb{<-} \hlkwd{relevel}\hlstd{(f0,} \hlstr{"II"}\hlstd{)}
\hlstd{f0}
\end{alltt}
\begin{verbatim}
##  [1] II  I   III III III I   I   II  I   II 
## Levels: II I III
\end{verbatim}
\end{kframe}
\end{knitrout}

  \end{itemize}
\end{frame}

\begin{frame}[fragile]
  \ft{Ordered Factors}
\begin{knitrout}\scriptsize
\definecolor{shadecolor}{rgb}{0.969, 0.969, 0.969}\color{fgcolor}\begin{kframe}
\begin{alltt}
\hlkwd{set.seed}\hlstd{(}\hlnum{433443534}\hlstd{)}
\hlstd{mon} \hlkwb{<-} \hlkwd{sample}\hlstd{(month.name,} \hlnum{29}\hlstd{,} \hlnum{TRUE}\hlstd{)}
\hlkwd{table}\hlstd{(}\hlkwd{factor}\hlstd{(mon))}
\end{alltt}
\begin{verbatim}
## 
##     April    August  December  February   January      July      June 
##         3         2         5         2         3         3         2 
##     March       May  November September 
##         1         3         3         2
\end{verbatim}
\end{kframe}
\end{knitrout}
\begin{knitrout}\scriptsize
\definecolor{shadecolor}{rgb}{0.969, 0.969, 0.969}\color{fgcolor}\begin{kframe}
\begin{alltt}
\hlstd{mon2} \hlkwb{<-} \hlkwd{factor}\hlstd{(mon,} \hlkwc{levels} \hlstd{= month.name,}
               \hlkwc{ordered} \hlstd{=} \hlnum{TRUE}\hlstd{)}
\hlkwd{table}\hlstd{(mon2)}
\end{alltt}
\begin{verbatim}
## mon2
##   January  February     March     April       May      June      July 
##         3         2         1         3         3         2         3 
##    August September   October  November  December 
##         2         2         0         3         5
\end{verbatim}
\end{kframe}
\end{knitrout}
{\small Order operator can be used with ordered factors}
\end{frame}

\begin{frame}[fragile]
  \ft{When Factors Are a PITA}
\begin{knitrout}\scriptsize
\definecolor{shadecolor}{rgb}{0.969, 0.969, 0.969}\color{fgcolor}\begin{kframe}
\begin{alltt}
\hlkwd{set.seed}\hlstd{(}\hlnum{423423}\hlstd{)}
\hlstd{ff} \hlkwb{<-} \hlkwd{factor}\hlstd{(}\hlkwd{sample}\hlstd{(}\hlnum{1}\hlopt{:}\hlnum{4}\hlstd{,} \hlnum{10}\hlstd{,} \hlnum{TRUE}\hlstd{))}
\end{alltt}
\end{kframe}
\end{knitrout}
\begin{knitrout}\scriptsize
\definecolor{shadecolor}{rgb}{0.969, 0.969, 0.969}\color{fgcolor}\begin{kframe}
\begin{alltt}
\hlkwd{mean}\hlstd{(ff)}
\end{alltt}


{\ttfamily\noindent\color{warningcolor}{\#\# Warning in mean.default(ff): argument is not numeric or logical: returning NA}}\begin{verbatim}
## [1] NA
\end{verbatim}
\begin{alltt}
\hlstd{ff} \hlopt{+} \hlnum{10}
\end{alltt}


{\ttfamily\noindent\color{warningcolor}{\#\# Warning in Ops.factor(ff, 10): '+' not meaningful for factors}}\begin{verbatim}
##  [1] NA NA NA NA NA NA NA NA NA NA
\end{verbatim}
\begin{alltt}
\hlkwd{c}\hlstd{(ff,} \hlnum{10}\hlstd{)} \hlcom{# Not a factor anymore}
\end{alltt}
\begin{verbatim}
##  [1]  1  2  1  1  2  1  3  1  2  3 10
\end{verbatim}
\end{kframe}
\end{knitrout}
\end{frame}
  
\begin{frame}[fragile]
  \ft{When Factors Are a PITA}
  
\begin{knitrout}\scriptsize
\definecolor{shadecolor}{rgb}{0.969, 0.969, 0.969}\color{fgcolor}\begin{kframe}
\begin{alltt}
\hlstd{(a} \hlkwb{<-} \hlkwd{factor}\hlstd{(}\hlkwd{sample}\hlstd{(letters,} \hlnum{10}\hlstd{,} \hlkwc{replace} \hlstd{=} \hlnum{TRUE}\hlstd{)))}
\end{alltt}
\begin{verbatim}
##  [1] a u o j i h d g n d
## Levels: a d g h i j n o u
\end{verbatim}
\begin{alltt}
\hlstd{(b} \hlkwb{<-} \hlkwd{factor}\hlstd{(}\hlkwd{sample}\hlstd{(letters,} \hlnum{10}\hlstd{,} \hlkwc{replace} \hlstd{=} \hlnum{TRUE}\hlstd{)))}
\end{alltt}
\begin{verbatim}
##  [1] t y k g v k p b d d
## Levels: b d g k p t v y
\end{verbatim}
\end{kframe}
\end{knitrout}
\begin{knitrout}\scriptsize
\definecolor{shadecolor}{rgb}{0.969, 0.969, 0.969}\color{fgcolor}\begin{kframe}
\begin{alltt}
\hlkwd{c}\hlstd{(a, b)}
\end{alltt}
\begin{verbatim}
##  [1] 1 9 8 6 5 4 2 3 7 2 6 8 4 3 7 4 5 1 2 2
\end{verbatim}
\end{kframe}
\end{knitrout}
\pause
\begin{knitrout}\scriptsize
\definecolor{shadecolor}{rgb}{0.969, 0.969, 0.969}\color{fgcolor}\begin{kframe}
\begin{alltt}
\hlkwd{factor}\hlstd{(}\hlkwd{c}\hlstd{(}\hlkwd{as.character}\hlstd{(a),} \hlkwd{as.character}\hlstd{(b)))}
\end{alltt}
\begin{verbatim}
##  [1] a u o j i h d g n d t y k g v k p b d d
## Levels: a b d g h i j k n o p t u v y
\end{verbatim}
\end{kframe}
\end{knitrout}
\end{frame}

\begin{frame}[fragile]
  \ft{Factors}
  \begin{itemize}
  \item Pros
    \begin{itemize}
    \item Needed for modelling categorical variable
    \item Memory efficient, i.e., factors only need to store values as
      integer and the unique levels as character strings
    \item Nice output
\begin{knitrout}\scriptsize
\definecolor{shadecolor}{rgb}{0.969, 0.969, 0.969}\color{fgcolor}\begin{kframe}
\begin{alltt}
\hlkwd{table}\hlstd{(}\hlkwd{factor}\hlstd{(}\hlkwd{c}\hlstd{(}\hlnum{1}\hlstd{,} \hlnum{2}\hlstd{,} \hlnum{3}\hlstd{),}
             \hlkwc{labels} \hlstd{=} \hlkwd{c}\hlstd{(}\hlstr{"Healthy"}\hlstd{,} \hlstr{"Diseased"}\hlstd{,} \hlstr{"Dead"}\hlstd{)))}
\end{alltt}
\begin{verbatim}
## 
##  Healthy Diseased     Dead 
##        1        1        1
\end{verbatim}
\end{kframe}
\end{knitrout}
    \end{itemize}
  \item Cons
    \begin{itemize}
    \item Require to be cautious for some data manipulation
    \end{itemize}
  \item I'd recommend reading data using the option
    \verb|stringsAsFactors=FALSE| and transform variables into factors
    as needed
  \end{itemize}
\end{frame}

\begin{frame}[fragile]
  \ft{Dates}
  
  R provides several options to deal with dates, which is a
  challenging problem, i.e., time zones, daylight savings, leap
  second, \dots
  
  \begin{itemize}
  \item \verb=as.Date= handles dates without time
  \item The {\bf chron} package handles dates and times, but without
    support for time zones
  \item The {\tt POSIXct} and {\tt POSIXlt} allow for dates and times
    with control for time zones
  \item The {\bf lubridate} packages is supposed to facilitate the use
    of dates and times in R
  \end{itemize}
  {\bf Rule of thumb:} Use the simplest technique possible. If you
  only have dates, use {\tt as.Date}
  
\end{frame}

\begin{frame}[fragile]
  \ft{Dates}
  \fst{{\tt as.Date}}
  
  \begin{itemize}
  \item {\tt as.Date} accepts a variety of input style through the
    {\tt format} argument
  \item Default is {\tt yyyy-mm-dd}
\begin{knitrout}\scriptsize
\definecolor{shadecolor}{rgb}{0.969, 0.969, 0.969}\color{fgcolor}\begin{kframe}
\begin{alltt}
\hlkwd{as.Date}\hlstd{(}\hlstr{"2014-06-12"}\hlstd{)}
\end{alltt}
\begin{verbatim}
## [1] "2014-06-12"
\end{verbatim}
\end{kframe}
\end{knitrout}

\begin{knitrout}\scriptsize
\definecolor{shadecolor}{rgb}{0.969, 0.969, 0.969}\color{fgcolor}\begin{kframe}
\begin{alltt}
\hlkwd{as.Date}\hlstd{(}\hlstr{"12.6.2014"}\hlstd{,} \hlkwc{format} \hlstd{=} \hlstr{"%d.%m.%Y"}\hlstd{)}
\end{alltt}
\begin{verbatim}
## [1] "2014-06-12"
\end{verbatim}
\end{kframe}
\end{knitrout}

\begin{knitrout}\scriptsize
\definecolor{shadecolor}{rgb}{0.969, 0.969, 0.969}\color{fgcolor}\begin{kframe}
\begin{alltt}
\hlkwd{as.Date}\hlstd{(}\hlstr{"12 June 14"}\hlstd{,} \hlkwc{format} \hlstd{=} \hlstr{"%d %B %y"}\hlstd{)}
\end{alltt}
\begin{verbatim}
## [1] "2014-06-12"
\end{verbatim}
\end{kframe}
\end{knitrout}

See {\tt ?strptime} for a complete list of format symbols
  \end{itemize}
  
\end{frame}

\begin{frame}[fragile]
  \ft{Dates}
  \fst{{\tt as.Date}}
  
  \begin{itemize}
  \item Internally, dates are stored as the number of days since
    January 1, 1970
  \item {\tt as.numeric} can be used to convert a date to its numeric form
\begin{knitrout}\small
\definecolor{shadecolor}{rgb}{0.969, 0.969, 0.969}\color{fgcolor}\begin{kframe}
\begin{alltt}
\hlkwd{as.integer}\hlstd{(}\hlkwd{as.Date}\hlstd{(}\hlstr{"2014-06-12"}\hlstd{))}
\end{alltt}
\begin{verbatim}
## [1] 16233
\end{verbatim}
\end{kframe}
\end{knitrout}
  \item The \verb=weekdays= and \verb=months= functions can be used to
    extract the dates' components
  \item Calculation on dates: See \verb=?Ops.Date=. Addition,
    subtraction, logical operations (\verb|==|, \verb=<=, \dots) are
    available

  \end{itemize}
\end{frame}

\begin{frame}[fragile]
  \ft{Dates}
  \fst{{\tt POSIXct} and {\tt POSIXlt}}
  
  \begin{itemize}
  \item \verb=POSIXct= is represented as seconds since January 1, 1970 GMT
  \item \verb=POSIXlt= is represented as a list
    \vspace{0.3cm}
    
    $\rightarrow$ Use \verb=POSIXct= for calculation and
    \verb=POSIXlt= for extracting date components
  \end{itemize}
  
\begin{knitrout}\scriptsize
\definecolor{shadecolor}{rgb}{0.969, 0.969, 0.969}\color{fgcolor}\begin{kframe}
\begin{alltt}
\hlstd{(t1} \hlkwb{<-} \hlkwd{as.POSIXct}\hlstd{(}\hlstr{"2014-06-12 10:15:00"}\hlstd{))}
\end{alltt}
\begin{verbatim}
## [1] "2014-06-12 10:15:00 CEST"
\end{verbatim}
\begin{alltt}
\hlstd{(t2} \hlkwb{<-} \hlkwd{as.POSIXlt}\hlstd{(}\hlstr{"2014-06-12 10:15:00"}\hlstd{))}
\end{alltt}
\begin{verbatim}
## [1] "2014-06-12 10:15:00 CEST"
\end{verbatim}
\end{kframe}
\end{knitrout}
\end{frame}

\begin{frame}[fragile]
  \ft{Dates}
  \fst{{\tt POSIXct} and {\tt POSIXlt}}
\begin{knitrout}\scriptsize
\definecolor{shadecolor}{rgb}{0.969, 0.969, 0.969}\color{fgcolor}\begin{kframe}
\begin{alltt}
\hlcom{## Internal representation of t2}
\hlkwd{str}\hlstd{(}\hlkwd{unclass}\hlstd{(t2))}
\end{alltt}
\begin{verbatim}
## List of 11
##  $ sec   : num 0
##  $ min   : int 15
##  $ hour  : int 10
##  $ mday  : int 12
##  $ mon   : int 5
##  $ year  : int 114
##  $ wday  : int 4
##  $ yday  : int 162
##  $ isdst : int 1
##  $ zone  : chr "CEST"
##  $ gmtoff: int NA
\end{verbatim}
\end{kframe}
\end{knitrout}
\end{frame}

\begin{frame}[fragile]
  \ft{Your Turn}
  
  Consider the data set {\tt data7.csv}
  \begin{enumerate}
  \item Read the data
  \item Compute the age of the patients
  \end{enumerate}

  
\end{frame}

\section{Text Processing}

\begin{xframe}
  \ft{Introduction}
  
  \begin{description}
  \item[Character:] A symbol in a written language, e.g, letters,
    numbers, punctuation marks, space, newlines, \dots
  \item[String:] A sequence of character bound together
  \end{description}
  Note that R does not distinguish between character and string
\begin{knitrout}\scriptsize
\definecolor{shadecolor}{rgb}{0.969, 0.969, 0.969}\color{fgcolor}\begin{kframe}
\begin{alltt}
\hlstd{test} \hlkwb{<-} \hlstr{"a"} \hlcom{# or 'a'}
\hlstd{test2} \hlkwb{<-} \hlstr{"apple"} \hlcom{# or 'apple'}
\hlkwd{class}\hlstd{(test)}
\end{alltt}
\begin{verbatim}
## [1] "character"
\end{verbatim}
\begin{alltt}
\hlkwd{class}\hlstd{(test2)}
\end{alltt}
\begin{verbatim}
## [1] "character"
\end{verbatim}
\end{kframe}
\end{knitrout}

\end{xframe}

\begin{xframe}
  \ft{Making Strings}
  
  Use single or double quotes to construct a string
\begin{knitrout}\scriptsize
\definecolor{shadecolor}{rgb}{0.969, 0.969, 0.969}\color{fgcolor}\begin{kframe}
\begin{alltt}
\hlstr{"cats"}
\end{alltt}
\begin{verbatim}
## [1] "cats"
\end{verbatim}
\begin{alltt}
\hlstr{'cats and dogs'}
\end{alltt}
\begin{verbatim}
## [1] "cats and dogs"
\end{verbatim}
\begin{alltt}
\hlstr{"cat's best friend"}
\end{alltt}
\begin{verbatim}
## [1] "cat's best friend"
\end{verbatim}
\end{kframe}
\end{knitrout}
Use {\tt nchar} to get the length of a string
\begin{knitrout}\scriptsize
\definecolor{shadecolor}{rgb}{0.969, 0.969, 0.969}\color{fgcolor}\begin{kframe}
\begin{alltt}
\hlkwd{length}\hlstd{(}\hlstr{"cat's best friend"}\hlstd{);} \hlkwd{nchar}\hlstd{(}\hlstr{"cat's best friend"}\hlstd{)}
\end{alltt}
\begin{verbatim}
## [1] 1
## [1] 17
\end{verbatim}
\end{kframe}
\end{knitrout}
 
\end{xframe}

\begin{xframe}
  \ft{Character-Valued Variables}
  
  Works just like other variables, e.g., 
\begin{knitrout}\scriptsize
\definecolor{shadecolor}{rgb}{0.969, 0.969, 0.969}\color{fgcolor}\begin{kframe}
\begin{alltt}
\hlcom{# A vector}
\hlstd{a} \hlkwb{<-} \hlkwd{c}\hlstd{(}\hlstr{"cats"}\hlstd{,} \hlstr{"and"}\hlstd{,} \hlstr{"dogs"}\hlstd{)}
\hlstd{a[}\hlnum{2}\hlstd{]}
\end{alltt}
\begin{verbatim}
## [1] "and"
\end{verbatim}
\begin{alltt}
\hlstd{a[}\hlkwd{c}\hlstd{(}\hlnum{1}\hlstd{,} \hlnum{3}\hlstd{)]}
\end{alltt}
\begin{verbatim}
## [1] "cats" "dogs"
\end{verbatim}
\end{kframe}
\end{knitrout}
Display characters
\begin{knitrout}\scriptsize
\definecolor{shadecolor}{rgb}{0.969, 0.969, 0.969}\color{fgcolor}\begin{kframe}
\begin{alltt}
\hlkwd{print}\hlstd{(a)}
\end{alltt}
\begin{verbatim}
## [1] "cats" "and"  "dogs"
\end{verbatim}
\begin{alltt}
\hlkwd{cat}\hlstd{(a)}
\end{alltt}
\begin{verbatim}
## cats and dogs
\end{verbatim}
\end{kframe}
\end{knitrout}
\end{xframe}

\begin{xframe}
  \ft{Substrings}
  
  The {\tt substr} permits to extract and/or replace substrings
\begin{knitrout}\scriptsize
\definecolor{shadecolor}{rgb}{0.969, 0.969, 0.969}\color{fgcolor}\begin{kframe}
\begin{alltt}
\hlcom{# Extract}
\hlstd{my_string} \hlkwb{<-} \hlstr{"cats don't like dogs"}
\hlkwd{substr}\hlstd{(my_string,} \hlkwc{start} \hlstd{=} \hlnum{6}\hlstd{,} \hlkwc{stop} \hlstd{=} \hlnum{15}\hlstd{)}
\end{alltt}
\begin{verbatim}
## [1] "don't like"
\end{verbatim}
\end{kframe}
\end{knitrout}
\begin{knitrout}\scriptsize
\definecolor{shadecolor}{rgb}{0.969, 0.969, 0.969}\color{fgcolor}\begin{kframe}
\begin{alltt}
\hlcom{# Works with vectors}
\hlstd{my_vector} \hlkwb{<-} \hlkwd{c}\hlstd{(}\hlstr{"cats"}\hlstd{,} \hlstr{"dogs"}\hlstd{,} \hlstr{"apple"}\hlstd{)}
\hlkwd{substr}\hlstd{(my_vector,} \hlnum{2}\hlstd{,} \hlnum{2}\hlstd{)}
\end{alltt}
\begin{verbatim}
## [1] "a" "o" "p"
\end{verbatim}
\end{kframe}
\end{knitrout}
\end{xframe}

\begin{xframe}
  \ft{Split Strings into Vectors}
  
  The {\tt strsplit} function permits to slit a string into a list
  containing multiple strings based on a give delimiter
\begin{knitrout}\scriptsize
\definecolor{shadecolor}{rgb}{0.969, 0.969, 0.969}\color{fgcolor}\begin{kframe}
\begin{alltt}
\hlstd{another_string} \hlkwb{<-} \hlstr{"cats, dogs and apples"}
\hlkwd{strsplit}\hlstd{(another_string,} \hlkwc{split} \hlstd{=} \hlstr{","}\hlstd{)}
\end{alltt}
\begin{verbatim}
## [[1]]
## [1] "cats"             " dogs and apples"
\end{verbatim}
\begin{alltt}
\hlkwd{strsplit}\hlstd{(another_string,} \hlkwc{split} \hlstd{=} \hlstr{" "}\hlstd{)[[}\hlnum{1}\hlstd{]]}
\end{alltt}
\begin{verbatim}
## [1] "cats,"  "dogs"   "and"    "apples"
\end{verbatim}
\end{kframe}
\end{knitrout}
\begin{knitrout}\scriptsize
\definecolor{shadecolor}{rgb}{0.969, 0.969, 0.969}\color{fgcolor}\begin{kframe}
\begin{alltt}
\hlstd{yet_another_string} \hlkwb{<-} \hlstr{"walk into a bar"}
\hlkwd{strsplit}\hlstd{(}\hlkwd{c}\hlstd{(another_string, yet_another_string),} \hlkwc{split} \hlstd{=} \hlstr{" "}\hlstd{)}
\end{alltt}
\begin{verbatim}
## [[1]]
## [1] "cats,"  "dogs"   "and"    "apples"
## 
## [[2]]
## [1] "walk" "into" "a"    "bar"
\end{verbatim}
\end{kframe}
\end{knitrout}

\end{xframe}

\begin{xframe}
  \ft{Build Strings from Multiple Parts}
  \fst{The {\tt paste} function}
  
  The {\tt paste} function combines multiple strings into a single
  strings. The {\tt sep} and {\tt collapse} arguments control the
  separation. 
\begin{knitrout}\scriptsize
\definecolor{shadecolor}{rgb}{0.969, 0.969, 0.969}\color{fgcolor}\begin{kframe}
\begin{alltt}
\hlkwd{paste}\hlstd{(}\hlkwd{c}\hlstd{(}\hlstr{"cats"}\hlstd{,} \hlstr{"dogs"}\hlstd{,} \hlstr{"apple"}\hlstd{),} \hlkwc{collapse} \hlstd{=} \hlstr{"|"}\hlstd{)} \hlcom{# BUT}
\end{alltt}
\begin{verbatim}
## [1] "cats|dogs|apple"
\end{verbatim}
\begin{alltt}
\hlkwd{paste}\hlstd{(}\hlkwd{c}\hlstd{(}\hlstr{"cats"}\hlstd{,} \hlstr{"dogs"}\hlstd{,} \hlstr{"apple"}\hlstd{),} \hlkwc{sep} \hlstd{=} \hlstr{"|"}\hlstd{)}
\end{alltt}
\begin{verbatim}
## [1] "cats"  "dogs"  "apple"
\end{verbatim}
\begin{alltt}
\hlcom{# collapse permits to concatenate strings from a single vector}
\end{alltt}
\end{kframe}
\end{knitrout}
\begin{knitrout}\scriptsize
\definecolor{shadecolor}{rgb}{0.969, 0.969, 0.969}\color{fgcolor}\begin{kframe}
\begin{alltt}
\hlkwd{paste}\hlstd{(}\hlstr{"cats"}\hlstd{,} \hlstr{"dogs"}\hlstd{,} \hlstr{"apple"}\hlstd{,} \hlkwc{sep} \hlstd{=} \hlstr{"|"}\hlstd{)}
\end{alltt}
\begin{verbatim}
## [1] "cats|dogs|apple"
\end{verbatim}
\end{kframe}
\end{knitrout}

\end{xframe}

\begin{xframe}
  \ft{Build Strings from Multiple Parts}
  
  A more complicated example: Estimate plus 95\% confidence
  interval
\begin{knitrout}\scriptsize
\definecolor{shadecolor}{rgb}{0.969, 0.969, 0.969}\color{fgcolor}\begin{kframe}
\begin{alltt}
\hlstd{est} \hlkwb{<-} \hlkwd{round}\hlstd{(}\hlnum{90.233982}\hlstd{,} \hlnum{2}\hlstd{)}
\hlstd{ci} \hlkwb{<-} \hlkwd{round}\hlstd{(}\hlkwd{c}\hlstd{(}\hlnum{40.48992}\hlstd{,} \hlnum{130.93774}\hlstd{),} \hlnum{2}\hlstd{)}
\hlkwd{paste}\hlstd{(est,} \hlstr{" (95% CI ["}\hlstd{, ci[}\hlnum{1}\hlstd{],} \hlstr{", "}\hlstd{, ci[}\hlnum{2}\hlstd{],} \hlstr{"])"}\hlstd{,} \hlkwc{sep} \hlstd{=} \hlstr{""}\hlstd{)}
\end{alltt}
\begin{verbatim}
## [1] "90.23 (95% CI [40.49, 130.94])"
\end{verbatim}
\end{kframe}
\end{knitrout}

\end{xframe}

\begin{xframe}
  \ft{Search and Replace}
  
  R provides several functions for searching and replacing text
  \begin{center}
    {\small
      \begin{tabular}{lp{8cm}} 
        \toprule
        {\tt grep}                 & Search for {\tt pattern} in a
        vector {\tt x} and return the indices of matches or matching
        string ({\tt value = TRUE})       \\[5pt]
        {\tt grepl} & As grep but returns a logical vector \\[5pt]
        {\tt regexpr} & Return character position of the first match
        as well as length of the match. -1 is returned if no match \\[5pt]
        {\tt gregexpr} & As {\tt regexpr} but reports all matches \\[5pt]
        {\tt regexec} & Comparable to {\tt regexpr} but returns a list\\[5pt]
        \midrule
        {\tt sub} & Finds pattern in text and replaces first
        match with specified string \\[5pt]
        {\tt gsub} & As {\tt sub} but replaces all matches\\[5pt]
        \bottomrule
      \end{tabular}}
  \end{center}
\end{xframe}

\begin{xframe}
  \ft{Simple Matching}
  
\begin{knitrout}\scriptsize
\definecolor{shadecolor}{rgb}{0.969, 0.969, 0.969}\color{fgcolor}\begin{kframe}
\begin{alltt}
\hlstd{l} \hlkwb{<-} \hlkwd{c}\hlstd{(}\hlstr{"apple"}\hlstd{,} \hlstr{"banana"}\hlstd{,} \hlstr{"grape"}\hlstd{,} \hlstr{"10"}\hlstd{,} \hlstr{"green.pepper"}\hlstd{)}
\hlkwd{grep}\hlstd{(}\hlkwc{pattern} \hlstd{=} \hlstr{"a"}\hlstd{,} \hlkwc{x} \hlstd{= l)}
\end{alltt}
\begin{verbatim}
## [1] 1 2 3
\end{verbatim}
\begin{alltt}
\hlkwd{grep}\hlstd{(}\hlkwc{pattern} \hlstd{=} \hlstr{"a"}\hlstd{,} \hlkwc{x} \hlstd{= l,} \hlkwc{value} \hlstd{=} \hlnum{TRUE}\hlstd{)}
\end{alltt}
\begin{verbatim}
## [1] "apple"  "banana" "grape"
\end{verbatim}
\begin{alltt}
\hlkwd{grepl}\hlstd{(}\hlstr{"a"}\hlstd{, l)}
\end{alltt}
\begin{verbatim}
## [1]  TRUE  TRUE  TRUE FALSE FALSE
\end{verbatim}
\begin{alltt}
\hlkwd{regexpr}\hlstd{(}\hlstr{"a"}\hlstd{, l)}
\end{alltt}
\begin{verbatim}
## [1]  1  2  3 -1 -1
## attr(,"match.length")
## [1]  1  1  1 -1 -1
## attr(,"useBytes")
## [1] TRUE
\end{verbatim}
\end{kframe}
\end{knitrout}

\end{xframe}

\begin{xframe}
  \ft{Simple Matching}

\begin{knitrout}\tiny
\definecolor{shadecolor}{rgb}{0.969, 0.969, 0.969}\color{fgcolor}\begin{kframe}
\begin{alltt}
\hlkwd{regexpr}\hlstd{(}\hlstr{"a"}\hlstd{, l)}
\end{alltt}
\begin{verbatim}
## [1]  1  2  3 -1 -1
## attr(,"match.length")
## [1]  1  1  1 -1 -1
## attr(,"useBytes")
## [1] TRUE
\end{verbatim}
\begin{alltt}
\hlkwd{gregexpr}\hlstd{(}\hlstr{"a"}\hlstd{, l)}
\end{alltt}
\begin{verbatim}
## [[1]]
## [1] 1
## attr(,"match.length")
## [1] 1
## attr(,"useBytes")
## [1] TRUE
## 
## [[2]]
## [1] 2 4 6
## attr(,"match.length")
## [1] 1 1 1
## attr(,"useBytes")
## [1] TRUE
## 
## [[3]]
## [1] 3
## attr(,"match.length")
## [1] 1
## attr(,"useBytes")
## [1] TRUE
## 
## [[4]]
## [1] -1
## attr(,"match.length")
## [1] -1
## attr(,"useBytes")
## [1] TRUE
## 
## [[5]]
## [1] -1
## attr(,"match.length")
## [1] -1
## attr(,"useBytes")
## [1] TRUE
\end{verbatim}
\begin{alltt}
\hlkwd{regexec}\hlstd{(}\hlstr{"a"}\hlstd{, l)}
\end{alltt}
\begin{verbatim}
## [[1]]
## [1] 1
## attr(,"match.length")
## [1] 1
## 
## [[2]]
## [1] 2
## attr(,"match.length")
## [1] 1
## 
## [[3]]
## [1] 3
## attr(,"match.length")
## [1] 1
## 
## [[4]]
## [1] -1
## attr(,"match.length")
## [1] -1
## 
## [[5]]
## [1] -1
## attr(,"match.length")
## [1] -1
\end{verbatim}
\end{kframe}
\end{knitrout}

\end{xframe}

\begin{xframe}
  \ft{Regular Expressions}
  
  A {\em regular expression} is a pattern that describes a set of
  strings. It is more or less a grammar of text searching
  
  \begin{itemize}
  \item {\tt .} represents any character
\begin{knitrout}\scriptsize
\definecolor{shadecolor}{rgb}{0.969, 0.969, 0.969}\color{fgcolor}\begin{kframe}
\begin{alltt}
\hlkwd{grep}\hlstd{(}\hlstr{"ap."}\hlstd{, l,} \hlkwc{value} \hlstd{=} \hlnum{TRUE}\hlstd{)}
\end{alltt}
\begin{verbatim}
## [1] "apple" "grape"
\end{verbatim}
\begin{alltt}
\hlkwd{grep}\hlstd{(}\hlstr{"\textbackslash{}\textbackslash{}.p"}\hlstd{, l,} \hlkwc{value} \hlstd{=} \hlnum{TRUE}\hlstd{)}
\end{alltt}
\begin{verbatim}
## [1] "green.pepper"
\end{verbatim}
\end{kframe}
\end{knitrout}
\item {\tt +} represents one or more occurrence
\begin{knitrout}\scriptsize
\definecolor{shadecolor}{rgb}{0.969, 0.969, 0.969}\color{fgcolor}\begin{kframe}
\begin{alltt}
\hlkwd{grep}\hlstd{(}\hlstr{"ba+"}\hlstd{, l,} \hlkwc{value} \hlstd{=} \hlnum{TRUE}\hlstd{)}
\end{alltt}
\begin{verbatim}
## [1] "banana"
\end{verbatim}
\begin{alltt}
\hlkwd{grep}\hlstd{(}\hlstr{"p+"}\hlstd{, l,} \hlkwc{value} \hlstd{=} \hlnum{TRUE}\hlstd{)}
\end{alltt}
\begin{verbatim}
## [1] "apple"        "grape"        "green.pepper"
\end{verbatim}
\end{kframe}
\end{knitrout}

  \end{itemize}
\end{xframe}

\begin{xframe}
  \ft{Regular Expressions}
  
  \begin{itemize}
  \item {\tt *} represents 0 or more occurrence
\begin{knitrout}\scriptsize
\definecolor{shadecolor}{rgb}{0.969, 0.969, 0.969}\color{fgcolor}\begin{kframe}
\begin{alltt}
\hlkwd{grep}\hlstd{(}\hlstr{"a*p"}\hlstd{, l,} \hlkwc{value} \hlstd{=} \hlnum{TRUE}\hlstd{)}
\end{alltt}
\begin{verbatim}
## [1] "apple"        "grape"        "green.pepper"
\end{verbatim}
\end{kframe}
\end{knitrout}
\item Group term with parentheses
\begin{knitrout}\scriptsize
\definecolor{shadecolor}{rgb}{0.969, 0.969, 0.969}\color{fgcolor}\begin{kframe}
\begin{alltt}
\hlkwd{grep}\hlstd{(}\hlstr{"(ap)+."}\hlstd{, l,} \hlkwc{value} \hlstd{=} \hlnum{TRUE}\hlstd{)}
\end{alltt}
\begin{verbatim}
## [1] "apple" "grape"
\end{verbatim}
\end{kframe}
\end{knitrout}
\item \verb=^= end of line; \verb=$= start of line
\begin{knitrout}\scriptsize
\definecolor{shadecolor}{rgb}{0.969, 0.969, 0.969}\color{fgcolor}\begin{kframe}
\begin{alltt}
\hlkwd{grep}\hlstd{(}\hlstr{"^g"}\hlstd{, l,} \hlkwc{value} \hlstd{=} \hlnum{TRUE}\hlstd{)}
\end{alltt}
\begin{verbatim}
## [1] "grape"        "green.pepper"
\end{verbatim}
\begin{alltt}
\hlkwd{grep}\hlstd{(}\hlstr{"e$"}\hlstd{, l,} \hlkwc{value} \hlstd{=} \hlnum{TRUE}\hlstd{)}
\end{alltt}
\begin{verbatim}
## [1] "apple" "grape"
\end{verbatim}
\end{kframe}
\end{knitrout}
  \end{itemize}
\end{xframe}

\begin{xframe}
  \ft{Regular Expressions}
  
  \begin{itemize}
  \item {\tt |} logical OR
\begin{knitrout}\scriptsize
\definecolor{shadecolor}{rgb}{0.969, 0.969, 0.969}\color{fgcolor}\begin{kframe}
\begin{alltt}
\hlkwd{grep}\hlstd{(}\hlstr{"(ap)|(an)"}\hlstd{, l,} \hlkwc{value} \hlstd{=} \hlnum{TRUE}\hlstd{)}
\end{alltt}
\begin{verbatim}
## [1] "apple"  "banana" "grape"
\end{verbatim}
\end{kframe}
\end{knitrout}
\item \verb=[ ]= creates special character classes, e.g.,
  \begin{itemize}
  \item \verb=[a-z]= lower case letters
  \item \verb=[a-gA-G0-5]= any of a to g, A to G, and 0 to 5
  \item \verb=[^0-9]= anything except a digit
  \end{itemize}
\begin{knitrout}\scriptsize
\definecolor{shadecolor}{rgb}{0.969, 0.969, 0.969}\color{fgcolor}\begin{kframe}
\begin{alltt}
\hlkwd{grep}\hlstd{(}\hlstr{"[a-c]"}\hlstd{, l,} \hlkwc{value} \hlstd{=} \hlnum{TRUE}\hlstd{)}
\end{alltt}
\begin{verbatim}
## [1] "apple"  "banana" "grape"
\end{verbatim}
\begin{alltt}
\hlkwd{grep}\hlstd{(}\hlstr{"[^a-z]"}\hlstd{, l,} \hlkwc{value} \hlstd{=} \hlnum{TRUE}\hlstd{)}
\end{alltt}
\begin{verbatim}
## [1] "10"           "green.pepper"
\end{verbatim}
\end{kframe}
\end{knitrout}

  \end{itemize}
\end{xframe}

\begin{xframe}
  \ft{Regular Expressions}
  
  \begin{itemize}
  \item \verb={n}= match exactly {\tt n} times
  \item \verb={n,}= match {\tt n} or more times
  \item \verb={n,m}= match at least {\tt n}  times but no more than
    {\tt m} times
  \end{itemize}
\begin{knitrout}\scriptsize
\definecolor{shadecolor}{rgb}{0.969, 0.969, 0.969}\color{fgcolor}\begin{kframe}
\begin{alltt}
\hlkwd{grep}\hlstd{(}\hlstr{"e\{2,\}"}\hlstd{, l,} \hlkwc{value} \hlstd{=} \hlnum{TRUE}\hlstd{)}
\end{alltt}
\begin{verbatim}
## [1] "green.pepper"
\end{verbatim}
\begin{alltt}
\hlkwd{grep}\hlstd{(}\hlstr{"p\{2\}.*r$"}\hlstd{, l,} \hlkwc{value} \hlstd{=} \hlnum{TRUE}\hlstd{)}
\end{alltt}
\begin{verbatim}
## [1] "green.pepper"
\end{verbatim}
\end{kframe}
\end{knitrout}

\end{xframe}

\begin{xframe}
  \ft{Your Turn}
  
  Consider the {\tt babynames} data in package {\bf babynames}
\begin{knitrout}\scriptsize
\definecolor{shadecolor}{rgb}{0.969, 0.969, 0.969}\color{fgcolor}\begin{kframe}
\begin{alltt}
\hlkwd{install.packages}\hlstd{(}\hlstr{"babynames"}\hlstd{)}
\hlkwd{require}\hlstd{(babynames)}
\hlstd{text} \hlkwb{<-} \hlkwd{unique}\hlstd{(babynames}\hlopt{$}\hlstd{name)}
\end{alltt}
\end{kframe}
\end{knitrout}
\begin{knitrout}\scriptsize
\definecolor{shadecolor}{rgb}{0.969, 0.969, 0.969}\color{fgcolor}\begin{kframe}


{\ttfamily\noindent\itshape\color{messagecolor}{\#\# Loading required package: babynames}}\end{kframe}
\end{knitrout}
Find in {\tt text}
\begin{enumerate}
\item All the names starting with L and with exactly 2 a's
\item All the names with at least 2 y's
\item All the names containing either 'Paris', 'London' or 'Berlin'
\end{enumerate}
\end{xframe}

\section{Data Manipulation}

\begin{frame}[fragile]
  \ft{Subscripting}
  
  \begin{itemize}
  \item Logical subscripts
\begin{knitrout}\scriptsize
\definecolor{shadecolor}{rgb}{0.969, 0.969, 0.969}\color{fgcolor}\begin{kframe}
\begin{alltt}
\hlstd{nums} \hlkwb{<-} \hlkwd{c}\hlstd{(}\hlnum{12}\hlstd{,} \hlnum{9}\hlstd{,} \hlnum{8}\hlstd{,} \hlnum{14}\hlstd{,} \hlnum{7}\hlstd{,} \hlnum{16}\hlstd{,} \hlnum{3}\hlstd{,} \hlnum{2}\hlstd{,} \hlnum{9}\hlstd{)}
\hlstd{nums} \hlopt{>} \hlnum{10}
\end{alltt}
\begin{verbatim}
## [1]  TRUE FALSE FALSE  TRUE FALSE  TRUE FALSE FALSE FALSE
\end{verbatim}
\end{kframe}
\end{knitrout}

\begin{knitrout}\scriptsize
\definecolor{shadecolor}{rgb}{0.969, 0.969, 0.969}\color{fgcolor}\begin{kframe}
\begin{alltt}
\hlstd{nums[nums} \hlopt{>} \hlnum{10}\hlstd{]}
\end{alltt}
\begin{verbatim}
## [1] 12 14 16
\end{verbatim}
\begin{alltt}
\hlkwd{which}\hlstd{(nums} \hlopt{>} \hlnum{10}\hlstd{)}
\end{alltt}
\begin{verbatim}
## [1] 1 4 6
\end{verbatim}
\begin{alltt}
\hlstd{nums[}\hlkwd{which}\hlstd{(nums} \hlopt{>} \hlnum{10}\hlstd{)]}
\end{alltt}
\begin{verbatim}
## [1] 12 14 16
\end{verbatim}
\end{kframe}
\end{knitrout}
 
\begin{knitrout}\scriptsize
\definecolor{shadecolor}{rgb}{0.969, 0.969, 0.969}\color{fgcolor}\begin{kframe}
\begin{alltt}
\hlstd{nums[nums} \hlopt{>} \hlnum{10}\hlstd{]} \hlkwb{<-} \hlnum{0}
\hlstd{nums}
\end{alltt}
\begin{verbatim}
## [1] 0 9 8 0 7 0 3 2 9
\end{verbatim}
\end{kframe}
\end{knitrout}

  \end{itemize}
  
\end{frame}

\begin{frame}[fragile]
  \ft{Subscripting Matrices and Arrays}
  
\begin{knitrout}\scriptsize
\definecolor{shadecolor}{rgb}{0.969, 0.969, 0.969}\color{fgcolor}\begin{kframe}
\begin{alltt}
\hlstd{(mat} \hlkwb{<-} \hlkwd{matrix}\hlstd{(}\hlnum{1}\hlopt{:}\hlnum{12}\hlstd{,} \hlnum{4}\hlstd{,} \hlnum{3}\hlstd{))}
\end{alltt}
\begin{verbatim}
##      [,1] [,2] [,3]
## [1,]    1    5    9
## [2,]    2    6   10
## [3,]    3    7   11
## [4,]    4    8   12
\end{verbatim}
\end{kframe}
\end{knitrout}

\begin{knitrout}\scriptsize
\definecolor{shadecolor}{rgb}{0.969, 0.969, 0.969}\color{fgcolor}\begin{kframe}
\begin{alltt}
\hlstd{mat[,} \hlkwd{c}\hlstd{(}\hlnum{3}\hlstd{,} \hlnum{1}\hlstd{)]}
\end{alltt}
\begin{verbatim}
##      [,1] [,2]
## [1,]    9    1
## [2,]   10    2
## [3,]   11    3
## [4,]   12    4
\end{verbatim}
\end{kframe}
\end{knitrout}

\end{frame}

\begin{frame}[fragile]
  \ft{Subscripting Matrices and Arrays}

\begin{knitrout}\scriptsize
\definecolor{shadecolor}{rgb}{0.969, 0.969, 0.969}\color{fgcolor}\begin{kframe}
\begin{alltt}
\hlstd{mat[,} \hlnum{1}\hlstd{]}
\end{alltt}
\begin{verbatim}
## [1] 1 2 3 4
\end{verbatim}
\end{kframe}
\end{knitrout}

\begin{knitrout}\scriptsize
\definecolor{shadecolor}{rgb}{0.969, 0.969, 0.969}\color{fgcolor}\begin{kframe}
\begin{alltt}
\hlstd{mat[,} \hlnum{1}\hlstd{,} \hlkwc{drop} \hlstd{=} \hlnum{FALSE}\hlstd{]}
\end{alltt}
\begin{verbatim}
##      [,1]
## [1,]    1
## [2,]    2
## [3,]    3
## [4,]    4
\end{verbatim}
\end{kframe}
\end{knitrout}

\begin{knitrout}\scriptsize
\definecolor{shadecolor}{rgb}{0.969, 0.969, 0.969}\color{fgcolor}\begin{kframe}
\begin{alltt}
\hlstd{mat[mat} \hlopt{>} \hlnum{4}\hlstd{]}
\end{alltt}
\begin{verbatim}
## [1]  5  6  7  8  9 10 11 12
\end{verbatim}
\end{kframe}
\end{knitrout}
\end{frame}

\begin{frame}[fragile]
  \ft{Lists}
  
  Lists are the most general R object. 
\begin{knitrout}\scriptsize
\definecolor{shadecolor}{rgb}{0.969, 0.969, 0.969}\color{fgcolor}\begin{kframe}
\begin{alltt}
\hlstd{(ll} \hlkwb{<-} \hlkwd{list}\hlstd{(}\hlkwc{a} \hlstd{=} \hlnum{1}\hlopt{:}\hlnum{3}\hlstd{,} \hlkwc{b} \hlstd{= month.name[}\hlnum{1}\hlopt{:}\hlnum{5}\hlstd{],} \hlkwc{c} \hlstd{=} \hlkwd{c}\hlstd{(}\hlnum{TRUE}\hlstd{,} \hlnum{FALSE}\hlstd{),}
            \hlkwc{d} \hlstd{=} \hlkwd{data.frame}\hlstd{(}\hlkwc{y} \hlstd{=} \hlkwd{rnorm}\hlstd{(}\hlnum{5}\hlstd{),} \hlkwc{x} \hlstd{=} \hlkwd{rbinom}\hlstd{(}\hlnum{5}\hlstd{,} \hlnum{1}\hlstd{,} \hlnum{.5}\hlstd{))))}
\end{alltt}
\begin{verbatim}
## $a
## [1] 1 2 3
## 
## $b
## [1] "January"  "February" "March"    "April"    "May"     
## 
## $c
## [1]  TRUE FALSE
## 
## $d
##            y x
## 1 -1.1065764 0
## 2  1.6957258 0
## 3 -1.0641906 1
## 4 -0.0415854 1
## 5  0.8534742 0
\end{verbatim}
\end{kframe}
\end{knitrout}
  
\end{frame}

\begin{frame}[fragile]
  \ft{Lists}

\begin{knitrout}\scriptsize
\definecolor{shadecolor}{rgb}{0.969, 0.969, 0.969}\color{fgcolor}\begin{kframe}
\begin{alltt}
\hlkwd{class}\hlstd{(ll[[}\hlnum{4}\hlstd{]]);} \hlkwd{class}\hlstd{(ll[[}\hlstr{"d"}\hlstd{]]);} \hlkwd{class}\hlstd{(ll}\hlopt{$}\hlstd{d)}
\end{alltt}
\begin{verbatim}
## [1] "data.frame"
## [1] "data.frame"
## [1] "data.frame"
\end{verbatim}
\begin{alltt}
\hlkwd{class}\hlstd{(ll[}\hlnum{4}\hlstd{])}
\end{alltt}
\begin{verbatim}
## [1] "list"
\end{verbatim}
\end{kframe}
\end{knitrout}

\begin{knitrout}\scriptsize
\definecolor{shadecolor}{rgb}{0.969, 0.969, 0.969}\color{fgcolor}\begin{kframe}
\begin{alltt}
\hlstd{ll[}\hlkwd{c}\hlstd{(}\hlnum{1}\hlstd{,} \hlnum{3}\hlstd{)]}
\end{alltt}
\begin{verbatim}
## $a
## [1] 1 2 3
## 
## $c
## [1]  TRUE FALSE
\end{verbatim}
\end{kframe}
\end{knitrout}
\end{frame}

\begin{frame}[fragile]
  \ft{Subscripting Data Frames}
  
\begin{knitrout}\scriptsize
\definecolor{shadecolor}{rgb}{0.969, 0.969, 0.969}\color{fgcolor}\begin{kframe}
\begin{alltt}
\hlkwd{set.seed}\hlstd{(}\hlnum{4234234}\hlstd{)}
\hlstd{(df} \hlkwb{<-} \hlkwd{data.frame}\hlstd{(}\hlkwc{x} \hlstd{=} \hlkwd{c}\hlstd{(}\hlkwd{rnorm}\hlstd{(}\hlnum{3}\hlstd{),} \hlnum{NA}\hlstd{,} \hlnum{3}\hlstd{),}
                 \hlkwc{y} \hlstd{=} \hlkwd{c}\hlstd{(}\hlnum{NA}\hlstd{,} \hlkwd{rexp}\hlstd{(}\hlnum{2}\hlstd{,} \hlnum{0.01}\hlstd{),} \hlnum{NA}\hlstd{,} \hlnum{3}\hlstd{)))}
\end{alltt}
\begin{verbatim}
##            x         y
## 1  1.7547348        NA
## 2 -0.3676785 108.34508
## 3 -1.5529115  85.43826
## 4         NA        NA
## 5  3.0000000   3.00000
\end{verbatim}
\begin{alltt}
\hlstd{df}\hlopt{$}\hlstd{x}
\end{alltt}
\begin{verbatim}
## [1]  1.7547348 -0.3676785 -1.5529115         NA  3.0000000
\end{verbatim}
\begin{alltt}
\hlstd{df[,} \hlstr{"x"}\hlstd{,} \hlkwc{drop} \hlstd{=} \hlnum{FALSE}\hlstd{]}
\end{alltt}
\begin{verbatim}
##            x
## 1  1.7547348
## 2 -0.3676785
## 3 -1.5529115
## 4         NA
## 5  3.0000000
\end{verbatim}
\end{kframe}
\end{knitrout}
\end{frame}

\begin{frame}[fragile]
  \ft{Subscripting Data Frames}

\begin{knitrout}\scriptsize
\definecolor{shadecolor}{rgb}{0.969, 0.969, 0.969}\color{fgcolor}\begin{kframe}
\begin{alltt}
\hlstd{df[df}\hlopt{$}\hlstd{y} \hlopt{>} \hlnum{10}\hlstd{, ]}
\end{alltt}
\begin{verbatim}
##               x         y
## NA           NA        NA
## 2    -0.3676785 108.34508
## 3    -1.5529115  85.43826
## NA.1         NA        NA
\end{verbatim}
\end{kframe}
\end{knitrout}

\begin{knitrout}\scriptsize
\definecolor{shadecolor}{rgb}{0.969, 0.969, 0.969}\color{fgcolor}\begin{kframe}
\begin{alltt}
\hlstd{df[}\hlopt{!}\hlkwd{is.na}\hlstd{(df}\hlopt{$}\hlstd{y)} \hlopt{&} \hlstd{df}\hlopt{$}\hlstd{y} \hlopt{>} \hlnum{10}\hlstd{, ]}
\end{alltt}
\begin{verbatim}
##            x         y
## 2 -0.3676785 108.34508
## 3 -1.5529115  85.43826
\end{verbatim}
\end{kframe}
\end{knitrout}

\begin{knitrout}\scriptsize
\definecolor{shadecolor}{rgb}{0.969, 0.969, 0.969}\color{fgcolor}\begin{kframe}
\begin{alltt}
\hlkwd{subset}\hlstd{(df, y} \hlopt{>} \hlnum{10}\hlstd{)}
\end{alltt}
\begin{verbatim}
##            x         y
## 2 -0.3676785 108.34508
## 3 -1.5529115  85.43826
\end{verbatim}
\end{kframe}
\end{knitrout}

\end{frame}

\begin{frame}[fragile]
  \ft{Subscripting Data Frames}

\begin{knitrout}\scriptsize
\definecolor{shadecolor}{rgb}{0.969, 0.969, 0.969}\color{fgcolor}\begin{kframe}
\begin{alltt}
\hlkwd{subset}\hlstd{(df, y} \hlopt{>} \hlnum{10}\hlstd{,} \hlkwc{select} \hlstd{= x)}
\end{alltt}
\begin{verbatim}
##            x
## 2 -0.3676785
## 3 -1.5529115
\end{verbatim}
\begin{alltt}
\hlkwd{subset}\hlstd{(df, y} \hlopt{>} \hlnum{10}\hlstd{,} \hlkwc{select} \hlstd{=} \hlnum{1}\hlstd{)}
\end{alltt}
\begin{verbatim}
##            x
## 2 -0.3676785
## 3 -1.5529115
\end{verbatim}
\end{kframe}
\end{knitrout}

{\small Order a data frame}
\begin{knitrout}\scriptsize
\definecolor{shadecolor}{rgb}{0.969, 0.969, 0.969}\color{fgcolor}\begin{kframe}
\begin{alltt}
\hlstd{df[}\hlkwd{order}\hlstd{(df}\hlopt{$}\hlstd{x), ]}
\end{alltt}
\begin{verbatim}
##            x         y
## 3 -1.5529115  85.43826
## 2 -0.3676785 108.34508
## 1  1.7547348        NA
## 5  3.0000000   3.00000
## 4         NA        NA
\end{verbatim}
\end{kframe}
\end{knitrout}

\end{frame}

\begin{frame}[fragile]
  \ft{Your Turn}
  \begin{enumerate}
  \item Compute a variable {\tt hypertension} that equals 1 if {\tt
      DIAS} $>120$ {\bf and} {\tt SYS} $>80$
  \item Compute a variable {\tt Hypotension} that equals 1 if {\tt
      DIAS} $<100$ {\bf and} {\tt SYS} $<65$
  \item Create a variable that equals 1 if the patient's {\tt CITY} is
    in New York state (NY)
  \item Create a data set that contains the patients that have
    hypotension or hypertension don't live in New York state
  \end{enumerate}
\end{frame}

\section{Data Aggregation}

\begin{frame}[fragile]
  \ft{Data Aggregation}
  
  \begin{itemize}
  \item For simple tabulation and cross-tabulation, the \verb=table=,
    \verb=ftable= and \verb=xtabs= functions are available
  \item For more complex tasks, the available functions can be
    classified into two groups
    \begin{itemize}
    \item Functions that operate on arrays and/or lists
      (e.g., \verb=*apply=, \verb=sweep=)
    \item Functions oriented towards data frames (e.g.,
      \verb=aggregate=, \verb=by=)
    \end{itemize}
  \end{itemize}
\end{frame}

\begin{frame}[fragile]
  \ft{The {\tt table} function}
  
\begin{knitrout}\scriptsize
\definecolor{shadecolor}{rgb}{0.969, 0.969, 0.969}\color{fgcolor}\begin{kframe}
\begin{alltt}
\hlkwd{data}\hlstd{(iris)}

\hlkwd{head}\hlstd{(iris)}
\end{alltt}
\begin{verbatim}
##   Sepal.Length Sepal.Width Petal.Length Petal.Width Species
## 1          5.1         3.5          1.4         0.2  setosa
## 2          4.9         3.0          1.4         0.2  setosa
## 3          4.7         3.2          1.3         0.2  setosa
## 4          4.6         3.1          1.5         0.2  setosa
## 5          5.0         3.6          1.4         0.2  setosa
## 6          5.4         3.9          1.7         0.4  setosa
\end{verbatim}
\end{kframe}
\end{knitrout}
\end{frame}

\begin{frame}[fragile]
  \ft{The {\tt table} function}
  
  \vspace{-0.5cm}
\begin{knitrout}\scriptsize
\definecolor{shadecolor}{rgb}{0.969, 0.969, 0.969}\color{fgcolor}\begin{kframe}
\begin{alltt}
\hlkwd{table}\hlstd{(iris}\hlopt{$}\hlstd{Species)}
\end{alltt}
\begin{verbatim}
## 
##     setosa versicolor  virginica 
##         50         50         50
\end{verbatim}
\end{kframe}
\end{knitrout}

\begin{knitrout}\scriptsize
\definecolor{shadecolor}{rgb}{0.969, 0.969, 0.969}\color{fgcolor}\begin{kframe}
\begin{alltt}
\hlkwd{table}\hlstd{(iris}\hlopt{$}\hlstd{Species, iris}\hlopt{$}\hlstd{Petal.Length} \hlopt{>} \hlnum{6}\hlstd{)}
\end{alltt}
\begin{verbatim}
##             
##              FALSE TRUE
##   setosa        50    0
##   versicolor    50    0
##   virginica     41    9
\end{verbatim}
\end{kframe}
\end{knitrout}

\begin{knitrout}\scriptsize
\definecolor{shadecolor}{rgb}{0.969, 0.969, 0.969}\color{fgcolor}\begin{kframe}
\begin{alltt}
\hlkwd{as.data.frame}\hlstd{(}\hlkwd{table}\hlstd{(iris}\hlopt{$}\hlstd{Species, iris}\hlopt{$}\hlstd{Petal.Length} \hlopt{>} \hlnum{6}\hlstd{))}
\end{alltt}
\begin{verbatim}
##         Var1  Var2 Freq
## 1     setosa FALSE   50
## 2 versicolor FALSE   50
## 3  virginica FALSE   41
## 4     setosa  TRUE    0
## 5 versicolor  TRUE    0
## 6  virginica  TRUE    9
\end{verbatim}
\end{kframe}
\end{knitrout}
\end{frame}

\begin{frame}[fragile]
  \ft{The {\tt table} function}

\begin{knitrout}\scriptsize
\definecolor{shadecolor}{rgb}{0.969, 0.969, 0.969}\color{fgcolor}\begin{kframe}
\begin{alltt}
\hlkwd{table}\hlstd{(iris}\hlopt{$}\hlstd{Species, iris}\hlopt{$}\hlstd{Petal.Length} \hlopt{>} \hlnum{6}\hlstd{, iris}\hlopt{$}\hlstd{Sepal.Width} \hlopt{>} \hlnum{3.5}\hlstd{)}
\end{alltt}
\begin{verbatim}
## , ,  = FALSE
## 
##             
##              FALSE TRUE
##   setosa        34    0
##   versicolor    50    0
##   virginica     41    6
## 
## , ,  = TRUE
## 
##             
##              FALSE TRUE
##   setosa        16    0
##   versicolor     0    0
##   virginica      0    3
\end{verbatim}
\end{kframe}
\end{knitrout}

\end{frame}

\begin{frame}[fragile]
  \ft{The {\tt table} function}

\begin{knitrout}\scriptsize
\definecolor{shadecolor}{rgb}{0.969, 0.969, 0.969}\color{fgcolor}\begin{kframe}
\begin{alltt}
\hlkwd{as.data.frame}\hlstd{(}\hlkwd{table}\hlstd{(iris}\hlopt{$}\hlstd{Species, iris}\hlopt{$}\hlstd{Petal.Length} \hlopt{>} \hlnum{6}\hlstd{, iris}\hlopt{$}\hlstd{Sepal.Width} \hlopt{>} \hlnum{3.5}\hlstd{))}
\end{alltt}
\begin{verbatim}
##          Var1  Var2  Var3 Freq
## 1      setosa FALSE FALSE   34
## 2  versicolor FALSE FALSE   50
## 3   virginica FALSE FALSE   41
## 4      setosa  TRUE FALSE    0
## 5  versicolor  TRUE FALSE    0
## 6   virginica  TRUE FALSE    6
## 7      setosa FALSE  TRUE   16
## 8  versicolor FALSE  TRUE    0
## 9   virginica FALSE  TRUE    0
## 10     setosa  TRUE  TRUE    0
## 11 versicolor  TRUE  TRUE    0
## 12  virginica  TRUE  TRUE    3
\end{verbatim}
\end{kframe}
\end{knitrout}

\end{frame}

\begin{frame}[fragile]
  \ft{The {\tt table} function}
  \fst{{\tt addmargins}}

\begin{knitrout}\scriptsize
\definecolor{shadecolor}{rgb}{0.969, 0.969, 0.969}\color{fgcolor}\begin{kframe}
\begin{alltt}
\hlstd{tt} \hlkwb{<-} \hlkwd{table}\hlstd{(iris}\hlopt{$}\hlstd{Species, iris}\hlopt{$}\hlstd{Petal.Length} \hlopt{>} \hlnum{6}\hlstd{)}

\hlkwd{addmargins}\hlstd{(tt,} \hlkwc{margin} \hlstd{=} \hlnum{1}\hlstd{)}
\end{alltt}
\begin{verbatim}
##             
##              FALSE TRUE
##   setosa        50    0
##   versicolor    50    0
##   virginica     41    9
##   Sum          141    9
\end{verbatim}
\begin{alltt}
\hlkwd{addmargins}\hlstd{(tt,} \hlkwc{margin} \hlstd{=} \hlkwd{c}\hlstd{(}\hlnum{1}\hlstd{,} \hlnum{2}\hlstd{))}
\end{alltt}
\begin{verbatim}
##             
##              FALSE TRUE Sum
##   setosa        50    0  50
##   versicolor    50    0  50
##   virginica     41    9  50
##   Sum          141    9 150
\end{verbatim}
\end{kframe}
\end{knitrout}
  
\end{frame}

\begin{frame}[fragile]
  \ft{The {\tt table} function}
  \fst{{\tt prop.table}}
  
\begin{knitrout}\scriptsize
\definecolor{shadecolor}{rgb}{0.969, 0.969, 0.969}\color{fgcolor}\begin{kframe}
\begin{alltt}
\hlkwd{prop.table}\hlstd{(tt,} \hlkwc{margin} \hlstd{=} \hlnum{1}\hlstd{)}
\end{alltt}
\begin{verbatim}
##             
##              FALSE TRUE
##   setosa      1.00 0.00
##   versicolor  1.00 0.00
##   virginica   0.82 0.18
\end{verbatim}
\begin{alltt}
\hlkwd{prop.table}\hlstd{(tt,} \hlkwc{margin} \hlstd{=} \hlnum{2}\hlstd{)}
\end{alltt}
\begin{verbatim}
##             
##                  FALSE      TRUE
##   setosa     0.3546099 0.0000000
##   versicolor 0.3546099 0.0000000
##   virginica  0.2907801 1.0000000
\end{verbatim}
\end{kframe}
\end{knitrout}

\end{frame}

\begin{frame}[fragile]
  \ft{The {\tt ftable} function}
  
  The \verb=ftable= function creates {\bf flat} tables
\begin{knitrout}\scriptsize
\definecolor{shadecolor}{rgb}{0.969, 0.969, 0.969}\color{fgcolor}\begin{kframe}
\begin{alltt}
\hlkwd{ftable}\hlstd{(iris}\hlopt{$}\hlstd{Species, iris}\hlopt{$}\hlstd{Petal.Length} \hlopt{>} \hlnum{6}\hlstd{, iris}\hlopt{$}\hlstd{Sepal.Width} \hlopt{>} \hlnum{3.5}\hlstd{)}
\end{alltt}
\begin{verbatim}
##                   FALSE TRUE
##                             
## setosa     FALSE     34   16
##            TRUE       0    0
## versicolor FALSE     50    0
##            TRUE       0    0
## virginica  FALSE     41    0
##            TRUE       6    3
\end{verbatim}
\end{kframe}
\end{knitrout}

\end{frame}

\begin{frame}[fragile]
  \ft{The {\tt xtabs} function}
  
  The \verb=xtabs= function produces similar results as the {\tt
    table} function but uses the {\tt formula} interface
\begin{knitrout}\scriptsize
\definecolor{shadecolor}{rgb}{0.969, 0.969, 0.969}\color{fgcolor}\begin{kframe}
\begin{alltt}
\hlstd{iris}\hlopt{$}\hlstd{sepal_width} \hlkwb{<-} \hlkwd{factor}\hlstd{(}\hlkwd{as.integer}\hlstd{(iris}\hlopt{$}\hlstd{Sepal.Width} \hlopt{>} \hlnum{3.5}\hlstd{),}
                           \hlkwc{levels} \hlstd{=} \hlkwd{c}\hlstd{(}\hlnum{0}\hlstd{,} \hlnum{1}\hlstd{),}
                           \hlkwc{labels} \hlstd{=} \hlkwd{c}\hlstd{(}\hlstr{"<= 3.5"}\hlstd{,} \hlstr{"> 3.5"}\hlstd{))}

\hlkwd{with}\hlstd{(iris,} \hlkwd{table}\hlstd{(Species, sepal_width))}
\end{alltt}
\begin{verbatim}
##             sepal_width
## Species      <= 3.5 > 3.5
##   setosa         34    16
##   versicolor     50     0
##   virginica      47     3
\end{verbatim}
\begin{alltt}
\hlkwd{xtabs}\hlstd{(}\hlopt{~} \hlstd{Species} \hlopt{+} \hlstd{sepal_width, iris)}
\end{alltt}
\begin{verbatim}
##             sepal_width
## Species      <= 3.5 > 3.5
##   setosa         34    16
##   versicolor     50     0
##   virginica      47     3
\end{verbatim}
\end{kframe}
\end{knitrout}

\end{frame}

\begin{frame}[fragile]
  \ft{Missing values with {\tt table}}
  \fst{The {\tt useNA} and {\tt exclude} Arguments}
  
\begin{knitrout}\scriptsize
\definecolor{shadecolor}{rgb}{0.969, 0.969, 0.969}\color{fgcolor}\begin{kframe}
\begin{alltt}
\hlstd{iris}\hlopt{$}\hlstd{sepal_widthNA} \hlkwb{<-} \hlstd{iris}\hlopt{$}\hlstd{sepal_width}
\hlstd{iris}\hlopt{$}\hlstd{sepal_widthNA[}\hlkwd{seq}\hlstd{(}\hlnum{1}\hlstd{,} \hlnum{150}\hlstd{,} \hlnum{25}\hlstd{)]} \hlkwb{<-} \hlnum{NA}

\hlkwd{with}\hlstd{(iris,} \hlkwd{table}\hlstd{(Species, sepal_widthNA))}
\end{alltt}
\begin{verbatim}
##             sepal_widthNA
## Species      <= 3.5 > 3.5
##   setosa         32    16
##   versicolor     48     0
##   virginica      45     3
\end{verbatim}
\begin{alltt}
\hlkwd{with}\hlstd{(iris,} \hlkwd{table}\hlstd{(Species, sepal_widthNA,} \hlkwc{useNA} \hlstd{=} \hlstr{"ifany"}\hlstd{))}
\end{alltt}
\begin{verbatim}
##             sepal_widthNA
## Species      <= 3.5 > 3.5 <NA>
##   setosa         32    16    2
##   versicolor     48     0    2
##   virginica      45     3    2
\end{verbatim}
\begin{alltt}
\hlcom{## If there is NAs, they will be included}
\end{alltt}
\end{kframe}
\end{knitrout}
\end{frame}

\begin{frame}[fragile]
  \ft{Missing values with {\tt table}}
  \fst{The {\tt useNA} and {\tt exclude} Arguments}
  \vspace{-0.4cm}
\begin{knitrout}\scriptsize
\definecolor{shadecolor}{rgb}{0.969, 0.969, 0.969}\color{fgcolor}\begin{kframe}
\begin{alltt}
\hlkwd{with}\hlstd{(iris,} \hlkwd{table}\hlstd{(Species, sepal_widthNA,} \hlkwc{useNA} \hlstd{=} \hlstr{"always"}\hlstd{))}
\end{alltt}
\begin{verbatim}
##             sepal_widthNA
## Species      <= 3.5 > 3.5 <NA>
##   setosa         32    16    2
##   versicolor     48     0    2
##   virginica      45     3    2
##   <NA>            0     0    0
\end{verbatim}
\begin{alltt}
\hlkwd{with}\hlstd{(iris,} \hlkwd{table}\hlstd{(Species, sepal_width,} \hlkwc{useNA} \hlstd{=} \hlstr{"always"}\hlstd{))}
\end{alltt}
\begin{verbatim}
##             sepal_width
## Species      <= 3.5 > 3.5 <NA>
##   setosa         34    16    0
##   versicolor     50     0    0
##   virginica      47     3    0
##   <NA>            0     0    0
\end{verbatim}
\begin{alltt}
\hlcom{## An NA column is always included, even if there is no missing values}
\end{alltt}
\end{kframe}
\end{knitrout}

The {\tt useNA} argument is specific to {\tt table}
\end{frame}

\begin{frame}[fragile]
  \ft{Missing values with {\tt table}}
  \fst{The {\tt useNA} and {\tt exclude} Arguments}
  The \verb=exclude= argument
  \begin{itemize}
  \item By default, \verb|exclude = c(NA, NaN)|
  \item E.g., \verb|exclude = NULL| to include missing values
  \end{itemize}

\begin{knitrout}\scriptsize
\definecolor{shadecolor}{rgb}{0.969, 0.969, 0.969}\color{fgcolor}\begin{kframe}
\begin{alltt}
\hlkwd{with}\hlstd{(iris,} \hlkwd{table}\hlstd{(Species, sepal_widthNA,} \hlkwc{exclude} \hlstd{=} \hlkwa{NULL}\hlstd{))}
\end{alltt}
\begin{verbatim}
##             sepal_widthNA
## Species      <= 3.5 > 3.5 <NA>
##   setosa         32    16    2
##   versicolor     48     0    2
##   virginica      45     3    2
##   <NA>            0     0    0
\end{verbatim}
\begin{alltt}
\hlcom{## equivalent to useNA = "always"}
\end{alltt}
\end{kframe}
\end{knitrout}

\end{frame}

\begin{frame}[fragile]
  \ft{Missing values with {\tt table}}
    \fst{The {\tt useNA} and {\tt exclude} Arguments}
\begin{knitrout}\scriptsize
\definecolor{shadecolor}{rgb}{0.969, 0.969, 0.969}\color{fgcolor}\begin{kframe}
\begin{alltt}
\hlkwd{with}\hlstd{(iris,} \hlkwd{table}\hlstd{(Species, sepal_widthNA,} \hlkwc{exclude} \hlstd{=} \hlstr{"setosa"}\hlstd{))}
\end{alltt}
\begin{verbatim}
##             sepal_widthNA
## Species      <= 3.5 > 3.5
##   versicolor     48     0
##   virginica      45     3
\end{verbatim}
\begin{alltt}
\hlkwd{with}\hlstd{(iris,} \hlkwd{table}\hlstd{(Species, sepal_widthNA,} \hlkwc{exclude} \hlstd{=} \hlstr{"setosa"}\hlstd{,} \hlkwc{useNA} \hlstd{=} \hlstr{"ifany"}\hlstd{))}
\end{alltt}
\begin{verbatim}
##             sepal_widthNA
## Species      <= 3.5 > 3.5 <NA>
##   versicolor     48     0    2
##   virginica      45     3    2
\end{verbatim}
\end{kframe}
\end{knitrout}
\begin{itemize}
\item \verb|exclude = NULL| is equivalent to \verb|useNA="always"|
\item \verb|exclude = "somethingElse"| only exclude level
  \verb="somethingElse"= from the factor
\end{itemize}
\end{frame}

\begin{frame}[fragile]
  \ft{Road Map for Aggregation}
  
  Three things to consider
  \begin{enumerate}
  \item How are the groups that divide the data defined?
  \item What is the nature of the data to be operated on?
  \item What is the desired end result
  \end{enumerate}
  
\end{frame}

\begin{frame}[fragile]
  \ft{Groups Defined as Lists Elements}
  
  \verb=sapply= or \verb=lapply= are the appropriate functions
  \begin{itemize}
  \item \verb=lapply= always returns a list
  \item \verb=sapply= tries to ``simplify'' the output
  \end{itemize}
\pause
\begin{knitrout}\tiny
\definecolor{shadecolor}{rgb}{0.969, 0.969, 0.969}\color{fgcolor}\begin{kframe}
\begin{alltt}
\hlstd{myList} \hlkwb{<-} \hlkwd{list}\hlstd{()}
\hlkwa{for} \hlstd{(i} \hlkwa{in} \hlnum{1}\hlopt{:}\hlnum{4}\hlstd{) \{}
    \hlstd{myList[[i]]} \hlkwb{<-} \hlkwd{rnorm}\hlstd{(}\hlkwc{n} \hlstd{=} \hlnum{3} \hlopt{*} \hlstd{i)}
\hlstd{\}}
\hlstd{myList}
\end{alltt}
\begin{verbatim}
## [[1]]
## [1]  0.3429579 -0.3193258  0.7808710
## 
## [[2]]
## [1]  1.16866312  0.01419804  0.45813283 -0.43180622  0.34224696 -1.30745260
## 
## [[3]]
## [1]  1.4005004 -1.7575754 -0.2415508  1.0928182 -1.1926425  1.8645074
## [7] -0.3128976  0.8755070 -1.9690762
## 
## [[4]]
##  [1]  1.1415283 -1.1269112 -2.2914106  0.9855559 -1.4959317  2.2773178
##  [7]  0.5160894 -0.5481570 -0.6098171 -1.4302086  1.2207910 -1.7708338
\end{verbatim}
\end{kframe}
\end{knitrout}

\end{frame}

\begin{frame}[fragile]
  \ft{Groups Defined as Lists Elements}

  {\small Both for \verb=lapply= and \verb=sapply=, the first argument
    is a list, the second argument is a function
    
    Third, fourth, \dots arguments are further arguments for the
    function that is applied}
\begin{knitrout}\scriptsize
\definecolor{shadecolor}{rgb}{0.969, 0.969, 0.969}\color{fgcolor}\begin{kframe}
\begin{alltt}
\hlkwd{lapply}\hlstd{(myList, length)}
\end{alltt}
\begin{verbatim}
## [[1]]
## [1] 3
## 
## [[2]]
## [1] 6
## 
## [[3]]
## [1] 9
## 
## [[4]]
## [1] 12
\end{verbatim}
\end{kframe}
\end{knitrout}


\begin{knitrout}\scriptsize
\definecolor{shadecolor}{rgb}{0.969, 0.969, 0.969}\color{fgcolor}\begin{kframe}
\begin{alltt}
\hlkwd{sapply}\hlstd{(myList, length)}
\end{alltt}
\begin{verbatim}
## [1]  3  6  9 12
\end{verbatim}
\end{kframe}
\end{knitrout}

\end{frame}

\begin{frame}[fragile]
  \ft{Groups Defined as Lists Elements}
  
\begin{knitrout}\scriptsize
\definecolor{shadecolor}{rgb}{0.969, 0.969, 0.969}\color{fgcolor}\begin{kframe}
\begin{alltt}
\hlstd{myList[[}\hlnum{2}\hlstd{]][}\hlkwd{c}\hlstd{(}\hlnum{3}\hlstd{,} \hlnum{5}\hlstd{)]} \hlkwb{<-} \hlnum{NA}
\hlkwd{sapply}\hlstd{(myList, mean)}
\end{alltt}
\begin{verbatim}
## [1]  0.26816768          NA -0.02671217 -0.26099896
\end{verbatim}
\end{kframe}
\end{knitrout}

\begin{knitrout}\scriptsize
\definecolor{shadecolor}{rgb}{0.969, 0.969, 0.969}\color{fgcolor}\begin{kframe}
\begin{alltt}
\hlkwd{sapply}\hlstd{(myList, mean,} \hlkwc{na.rm} \hlstd{=} \hlnum{TRUE}\hlstd{)}
\end{alltt}
\begin{verbatim}
## [1]  0.26816768 -0.13909942 -0.02671217 -0.26099896
\end{verbatim}
\end{kframe}
\end{knitrout}

\begin{knitrout}\scriptsize
\definecolor{shadecolor}{rgb}{0.969, 0.969, 0.969}\color{fgcolor}\begin{kframe}
\begin{alltt}
\hlkwd{sapply}\hlstd{(myList, quantile,} \hlkwc{probs} \hlstd{=} \hlkwd{c}\hlstd{(}\hlnum{0.25}\hlstd{,} \hlnum{0.75}\hlstd{),} \hlkwc{na.rm} \hlstd{=} \hlnum{TRUE}\hlstd{)}
\end{alltt}
\begin{verbatim}
##          [,1]       [,2]      [,3]      [,4]
## 25% 0.0118160 -0.6507178 -1.192642 -1.446639
## 75% 0.5619144  0.3028143  1.092818  1.024549
\end{verbatim}
\end{kframe}
\end{knitrout}

\end{frame}

\begin{frame}[fragile]
  \ft{Groups Defined as Lists Elements}

\begin{knitrout}\tiny
\definecolor{shadecolor}{rgb}{0.969, 0.969, 0.969}\color{fgcolor}\begin{kframe}
\begin{alltt}
\hlcom{## A user defined function}
\hlkwd{lapply}\hlstd{(myList,} \hlkwa{function}\hlstd{(}\hlkwc{x}\hlstd{) \{}
    \hlkwd{data.frame}\hlstd{(}
        \hlkwc{Mean} \hlstd{=} \hlkwd{mean}\hlstd{(x,} \hlkwc{na.rm} \hlstd{=} \hlnum{TRUE}\hlstd{),}
        \hlkwc{SD} \hlstd{=} \hlkwd{sd}\hlstd{(x,} \hlkwc{na.rm} \hlstd{=} \hlnum{TRUE}\hlstd{),}
        \hlkwc{Min} \hlstd{=} \hlkwd{min}\hlstd{(x,} \hlkwc{na.rm} \hlstd{=} \hlnum{TRUE}\hlstd{),}
        \hlkwc{Max} \hlstd{=} \hlkwd{max}\hlstd{(x,} \hlkwc{na.rm} \hlstd{=} \hlnum{TRUE}\hlstd{))}
\hlstd{\})}
\end{alltt}
\begin{verbatim}
## [[1]]
##        Mean        SD        Min      Max
## 1 0.2681677 0.5538984 -0.3193258 0.780871
## 
## [[2]]
##         Mean       SD       Min      Max
## 1 -0.1390994 1.030286 -1.307453 1.168663
## 
## [[3]]
##          Mean       SD       Min      Max
## 1 -0.02671217 1.411432 -1.969076 1.864507
## 
## [[4]]
##        Mean       SD       Min      Max
## 1 -0.260999 1.446369 -2.291411 2.277318
\end{verbatim}
\end{kframe}
\end{knitrout}

\end{frame}

\begin{frame}[fragile]
  \ft{Groups Defined as Lists Elements}

\begin{knitrout}\scriptsize
\definecolor{shadecolor}{rgb}{0.969, 0.969, 0.969}\color{fgcolor}\begin{kframe}
\begin{alltt}
\hlcom{## A user defined function}
\hlkwd{sapply}\hlstd{(myList,} \hlkwa{function}\hlstd{(}\hlkwc{x}\hlstd{) \{}
    \hlkwd{data.frame}\hlstd{(}
        \hlkwc{Mean} \hlstd{=} \hlkwd{mean}\hlstd{(x,} \hlkwc{na.rm} \hlstd{=} \hlnum{TRUE}\hlstd{),}
        \hlkwc{SD} \hlstd{=} \hlkwd{sd}\hlstd{(x,} \hlkwc{na.rm} \hlstd{=} \hlnum{TRUE}\hlstd{),}
        \hlkwc{Min} \hlstd{=} \hlkwd{min}\hlstd{(x,} \hlkwc{na.rm} \hlstd{=} \hlnum{TRUE}\hlstd{),}
        \hlkwc{Max} \hlstd{=} \hlkwd{max}\hlstd{(x,} \hlkwc{na.rm} \hlstd{=} \hlnum{TRUE}\hlstd{))}
\hlstd{\})}
\end{alltt}
\begin{verbatim}
##      [,1]       [,2]       [,3]        [,4]     
## Mean 0.2681677  -0.1390994 -0.02671217 -0.260999
## SD   0.5538984  1.030286   1.411432    1.446369 
## Min  -0.3193258 -1.307453  -1.969076   -2.291411
## Max  0.780871   1.168663   1.864507    2.277318
\end{verbatim}
\end{kframe}
\end{knitrout}

\end{frame}

\begin{frame}[fragile]
  \ft{Groups Defined as Lists Elements}
  
\begin{knitrout}\scriptsize
\definecolor{shadecolor}{rgb}{0.969, 0.969, 0.969}\color{fgcolor}\begin{kframe}
\begin{alltt}
\hlstd{mySummary} \hlkwb{<-} \hlkwa{function}\hlstd{(}\hlkwc{x}\hlstd{,} \hlkwc{na.rm} \hlstd{=} \hlnum{FALSE}\hlstd{) \{}
    \hlkwd{data.frame}\hlstd{(}
        \hlkwc{Mean} \hlstd{=} \hlkwd{mean}\hlstd{(x,} \hlkwc{na.rm} \hlstd{= na.rm),}
        \hlkwc{SD} \hlstd{=} \hlkwd{sd}\hlstd{(x,} \hlkwc{na.rm} \hlstd{= na.rm),}
        \hlkwc{Min} \hlstd{=} \hlkwd{min}\hlstd{(x,} \hlkwc{na.rm} \hlstd{= na.rm),}
        \hlkwc{Max} \hlstd{=} \hlkwd{max}\hlstd{(x,} \hlkwc{na.rm} \hlstd{= na.rm))}
\hlstd{\}}

\hlkwd{sapply}\hlstd{(myList, mySummary,} \hlkwc{na.rm} \hlstd{=} \hlnum{TRUE}\hlstd{)}
\end{alltt}
\begin{verbatim}
##      [,1]       [,2]       [,3]        [,4]     
## Mean 0.2681677  -0.1390994 -0.02671217 -0.260999
## SD   0.5538984  1.030286   1.411432    1.446369 
## Min  -0.3193258 -1.307453  -1.969076   -2.291411
## Max  0.780871   1.168663   1.864507    2.277318
\end{verbatim}
\end{kframe}
\end{knitrout}
\end{frame}

\begin{frame}[fragile]
  \ft{Groups Defined as Lists Elements}
  
  \verb=sapply= or \verb=lapply= can be used as alternative to
  loops. This way you don't have to take care too much of the form of
  the output
\begin{knitrout}\scriptsize
\definecolor{shadecolor}{rgb}{0.969, 0.969, 0.969}\color{fgcolor}\begin{kframe}
\begin{alltt}
\hlcom{## check type 1 error of the t-test}
\hlstd{check_level} \hlkwb{<-} \hlkwa{function}\hlstd{(}\hlkwc{i}\hlstd{,} \hlkwc{n} \hlstd{=} \hlnum{100}\hlstd{) \{}
    \hlstd{a} \hlkwb{<-} \hlkwd{rnorm}\hlstd{(n)}
    \hlstd{b} \hlkwb{<-} \hlkwd{rnorm}\hlstd{(n)}
    \hlstd{tt} \hlkwb{<-} \hlkwd{t.test}\hlstd{(a, b)}
    \hlstd{tt}\hlopt{$}\hlstd{p.value} \hlopt{<} \hlnum{0.05}
\hlstd{\}}
\end{alltt}
\end{kframe}
\end{knitrout}

\begin{knitrout}\scriptsize
\definecolor{shadecolor}{rgb}{0.969, 0.969, 0.969}\color{fgcolor}\begin{kframe}
\begin{alltt}
\hlstd{nsimul} \hlkwb{<-} \hlnum{1000}
\hlstd{res} \hlkwb{<-} \hlkwd{sapply}\hlstd{(}\hlnum{1}\hlopt{:}\hlstd{nsimul, check_level,} \hlkwc{n} \hlstd{=} \hlnum{100}\hlstd{)}
\hlkwd{sum}\hlstd{(res)} \hlopt{/} \hlstd{nsimul}
\end{alltt}
\begin{verbatim}
## [1] 0.047
\end{verbatim}
\end{kframe}
\end{knitrout}

\end{frame}

\begin{frame}[fragile]
  \ft{Groups Defined as Lists Elements}

  For this kind of simple repetitive tasks, the \verb=replicate=
  function can also be used
  
\begin{knitrout}\scriptsize
\definecolor{shadecolor}{rgb}{0.969, 0.969, 0.969}\color{fgcolor}\begin{kframe}
\begin{alltt}
\hlstd{res2} \hlkwb{<-} \hlkwd{replicate}\hlstd{(nsimul,} \hlkwd{t.test}\hlstd{(}\hlkwd{rnorm}\hlstd{(}\hlnum{10}\hlstd{),} \hlkwd{rnorm}\hlstd{(}\hlnum{10}\hlstd{))}\hlopt{$}\hlstd{p.value} \hlopt{<} \hlnum{0.05}\hlstd{)}
\hlkwd{sum}\hlstd{(res2)} \hlopt{/} \hlstd{nsimul}
\end{alltt}
\begin{verbatim}
## [1] 0.044
\end{verbatim}
\end{kframe}
\end{knitrout}
\begin{itemize}
\item First argument is the number of replication
\item Second argument is an {\em expression}, i.e., a piece of R
  language and not a function 
\end{itemize}
\end{frame}

\begin{frame}[fragile]
  \ft{Groups Defined by Rows or Columns of a Matrix/Array}
  
  In this case, the {\tt apply} function is the logical choice. 
  
  The \verb=apply= function requires three arguments
  \begin{itemize}
  \item the array/matrix on which to operate
  \item An index telling \verb=apply= which dimension to operate on (\verb=1=
    on rows; \verb=2= on columns, \verb=c(1, 2)= on both
  \item The function to use
  \item Optionally further arguments to be used by the function that
    we want to apply
  \end{itemize}
 
\begin{knitrout}\scriptsize
\definecolor{shadecolor}{rgb}{0.969, 0.969, 0.969}\color{fgcolor}\begin{kframe}
\begin{alltt}
\hlkwd{apply}\hlstd{(iris[,} \hlnum{1}\hlopt{:}\hlnum{4}\hlstd{],} \hlnum{2}\hlstd{, mean)}
\end{alltt}
\begin{verbatim}
## Sepal.Length  Sepal.Width Petal.Length  Petal.Width 
##     5.843333     3.057333     3.758000     1.199333
\end{verbatim}
\end{kframe}
\end{knitrout}
\end{frame}

\begin{frame}[fragile]
  \ft{Groups Defined by Rows or Columns of a Matrix/Array}

\begin{knitrout}\scriptsize
\definecolor{shadecolor}{rgb}{0.969, 0.969, 0.969}\color{fgcolor}\begin{kframe}
\begin{alltt}
\hlkwd{apply}\hlstd{(iris[,} \hlnum{1}\hlopt{:}\hlnum{4}\hlstd{],} \hlnum{2}\hlstd{, mySummary)}
\end{alltt}
\begin{verbatim}
## $Sepal.Length
##       Mean        SD Min Max
## 1 5.843333 0.8280661 4.3 7.9
## 
## $Sepal.Width
##       Mean        SD Min Max
## 1 3.057333 0.4358663   2 4.4
## 
## $Petal.Length
##    Mean       SD Min Max
## 1 3.758 1.765298   1 6.9
## 
## $Petal.Width
##       Mean        SD Min Max
## 1 1.199333 0.7622377 0.1 2.5
\end{verbatim}
\end{kframe}
\end{knitrout}

\end{frame}

\begin{frame}[fragile]
  \ft{Groups Defined by Rows or Columns of a Matrix/Array}
  \fst{{\tt rowSums}, {\tt colSums}, {\tt rowMeans}, {\tt colMeans}}
  
  These are specialised functions that are potentially way faster than
  {\tt apply} (which is a general function)

\begin{knitrout}\scriptsize
\definecolor{shadecolor}{rgb}{0.969, 0.969, 0.969}\color{fgcolor}\begin{kframe}
\begin{alltt}
\hlkwd{colMeans}\hlstd{(iris[,} \hlnum{1}\hlopt{:}\hlnum{4}\hlstd{],} \hlkwc{na.rm} \hlstd{=} \hlnum{TRUE}\hlstd{)}
\end{alltt}
\begin{verbatim}
## Sepal.Length  Sepal.Width Petal.Length  Petal.Width 
##     5.843333     3.057333     3.758000     1.199333
\end{verbatim}
\end{kframe}
\end{knitrout}

\begin{knitrout}\scriptsize
\definecolor{shadecolor}{rgb}{0.969, 0.969, 0.969}\color{fgcolor}\begin{kframe}
\begin{alltt}
\hlkwd{colSums}\hlstd{(iris[,} \hlnum{1}\hlopt{:}\hlnum{4}\hlstd{]} \hlopt{>} \hlnum{2}\hlstd{,} \hlkwc{na.rm} \hlstd{=} \hlnum{TRUE}\hlstd{)}
\end{alltt}
\begin{verbatim}
## Sepal.Length  Sepal.Width Petal.Length  Petal.Width 
##          150          149          100           23
\end{verbatim}
\end{kframe}
\end{knitrout}

\end{frame}

\begin{frame}[fragile]
  \ft{Groups Based on One or More Grouping Variables}
  
  A very common operation
  
  A lot of choice in base R + a couple of additional packages that
  facilitates these operations
  
  \begin{itemize}
  \item \verb=aggregate=
  \item \verb=tapply=, \verb=by=
  \item {\em split-apply-combine} strategy
    \begin{itemize}
    \item \verb=split=, \verb=lapply=, \verb=do.call=
    \item {\bf plyr}, {\bf dplyr} package
    \item \dots
    \end{itemize}

  \end{itemize}
\end{frame}

\begin{frame}[fragile]
  \ft{Groups Based on One or More Grouping Variables}
  \fst{{\tt aggregate}}
  
  A natural choice for data summaries of several variables
  \begin{itemize}
  \item First argument: A formula
    \begin{itemize}
    \item LHS: Variables to ``summarise''
    \item RHS: Grouping variables
    \end{itemize}
  \item Second argument: A data frame
  \item Third argument: Function to apply
  \item \dots; Further arguments for {\tt FUN} 
  \end{itemize}

\begin{knitrout}\scriptsize
\definecolor{shadecolor}{rgb}{0.969, 0.969, 0.969}\color{fgcolor}\begin{kframe}
\begin{alltt}
\hlkwd{aggregate}\hlstd{(}\hlkwd{cbind}\hlstd{(Sepal.Length, Sepal.Width)} \hlopt{~} \hlstd{Species, iris, mean)}
\end{alltt}
\begin{verbatim}
##      Species Sepal.Length Sepal.Width
## 1     setosa        5.006       3.428
## 2 versicolor        5.936       2.770
## 3  virginica        6.588       2.974
\end{verbatim}
\end{kframe}
\end{knitrout}
\end{frame}

\begin{frame}[fragile]
  \ft{Groups Based on One or More Grouping Variables}
  \fst{{\tt aggregate}}
  
\begin{knitrout}\scriptsize
\definecolor{shadecolor}{rgb}{0.969, 0.969, 0.969}\color{fgcolor}\begin{kframe}
\begin{alltt}
\hlstd{iris}\hlopt{$}\hlstd{Petal.Length.f} \hlkwb{<-} \hlkwd{factor}\hlstd{(iris}\hlopt{$}\hlstd{Petal.Length} \hlopt{>} \hlnum{4.8}\hlstd{,}
                              \hlkwc{levels} \hlstd{=} \hlkwd{c}\hlstd{(}\hlnum{FALSE}\hlstd{,} \hlnum{TRUE}\hlstd{),}
                              \hlkwc{labels} \hlstd{=} \hlkwd{c}\hlstd{(}\hlstr{"Small petals"}\hlstd{,} \hlstr{"Big petals"}\hlstd{))}
\hlkwd{aggregate}\hlstd{(}\hlkwd{cbind}\hlstd{(Sepal.Length, Sepal.Width)} \hlopt{~} \hlstd{Species} \hlopt{+} \hlstd{Petal.Length.f,}
          \hlkwc{data} \hlstd{= iris,} \hlkwc{FUN} \hlstd{= mean)}
\end{alltt}
\begin{verbatim}
##      Species Petal.Length.f Sepal.Length Sepal.Width
## 1     setosa   Small petals     5.006000    3.428000
## 2 versicolor   Small petals     5.889130    2.765217
## 3  virginica   Small petals     5.700000    2.766667
## 4 versicolor     Big petals     6.475000    2.825000
## 5  virginica     Big petals     6.644681    2.987234
\end{verbatim}
\end{kframe}
\end{knitrout}

\end{frame}

\begin{frame}[fragile]
  \ft{Groups Based on One or More Grouping Variables}
  \fst{{\tt tapply}}
  
  Returns an array with as many dimensions as there were vectors that
  defined the groups, but can only process a single vector
  
\begin{knitrout}\scriptsize
\definecolor{shadecolor}{rgb}{0.969, 0.969, 0.969}\color{fgcolor}\begin{kframe}
\begin{alltt}
\hlkwd{with}\hlstd{(iris,} \hlkwd{tapply}\hlstd{(}\hlkwc{X} \hlstd{= Sepal.Length,}
                  \hlkwc{INDEX} \hlstd{=} \hlkwd{list}\hlstd{(Species, Petal.Length.f),}
                  \hlkwc{FUN} \hlstd{= mean))}
\end{alltt}
\begin{verbatim}
##            Small petals Big petals
## setosa          5.00600         NA
## versicolor      5.88913   6.475000
## virginica       5.70000   6.644681
\end{verbatim}
\end{kframe}
\end{knitrout}

\end{frame}

\begin{frame}[fragile]
  \ft{Groups Based on One or More Grouping Variables}
  \fst{{\tt tapply}}

  Also works if FUN does not return a scalar
\begin{knitrout}\scriptsize
\definecolor{shadecolor}{rgb}{0.969, 0.969, 0.969}\color{fgcolor}\begin{kframe}
\begin{alltt}
\hlkwd{with}\hlstd{(iris,} \hlkwd{tapply}\hlstd{(}\hlkwc{X} \hlstd{= Sepal.Length,}
                  \hlkwc{INDEX} \hlstd{= Species,}
                  \hlkwc{FUN} \hlstd{= range))}
\end{alltt}
\begin{verbatim}
## $setosa
## [1] 4.3 5.8
## 
## $versicolor
## [1] 4.9 7.0
## 
## $virginica
## [1] 4.9 7.9
\end{verbatim}
\end{kframe}
\end{knitrout}
\end{frame}

\begin{frame}[fragile]
  \ft{Groups Based on One or More Grouping Variables}
  \fst{{\tt tapply}}

\begin{knitrout}\scriptsize
\definecolor{shadecolor}{rgb}{0.969, 0.969, 0.969}\color{fgcolor}\begin{kframe}
\begin{alltt}
\hlstd{(tt} \hlkwb{<-} \hlkwd{with}\hlstd{(iris,} \hlkwd{tapply}\hlstd{(}\hlkwc{X} \hlstd{= Sepal.Length,}
                        \hlkwc{INDEX} \hlstd{=} \hlkwd{list}\hlstd{(Species, Petal.Length.f),}
                        \hlkwc{FUN} \hlstd{= range)))}
\end{alltt}
\begin{verbatim}
##            Small petals Big petals
## setosa     Numeric,2    NULL      
## versicolor Numeric,2    Numeric,2 
## virginica  Numeric,2    Numeric,2
\end{verbatim}
\end{kframe}
\end{knitrout}
In this case, a matrix of lists is returned...

\end{frame}

\begin{frame}[fragile]
  \ft{Groups Based on One or More Grouping Variables}
  \fst{{\tt tapply}}

  But individual elements can still be accessed
\begin{knitrout}\scriptsize
\definecolor{shadecolor}{rgb}{0.969, 0.969, 0.969}\color{fgcolor}\begin{kframe}
\begin{alltt}
\hlstd{tt[[}\hlstr{"setosa"}\hlstd{,} \hlstr{"Small petals"}\hlstd{]]}
\end{alltt}
\begin{verbatim}
## [1] 4.3 5.8
\end{verbatim}
\end{kframe}
\end{knitrout}

\begin{knitrout}\scriptsize
\definecolor{shadecolor}{rgb}{0.969, 0.969, 0.969}\color{fgcolor}\begin{kframe}
\begin{alltt}
\hlkwd{str}\hlstd{(tt)}
\end{alltt}
\begin{verbatim}
## List of 6
##  $ : num [1:2] 4.3 5.8
##  $ : num [1:2] 4.9 7
##  $ : num [1:2] 4.9 6.2
##  $ : NULL
##  $ : num [1:2] 6 6.9
##  $ : num [1:2] 5.6 7.9
##  - attr(*, "dim")= int [1:2] 3 2
##  - attr(*, "dimnames")=List of 2
##   ..$ : chr [1:3] "setosa" "versicolor" "virginica"
##   ..$ : chr [1:2] "Small petals" "Big petals"
\end{verbatim}
\end{kframe}
\end{knitrout}

\end{frame}

\begin{frame}[fragile]
  \ft{Groups Based on One or More Grouping Variables}
  \fst{{\tt by}}
  
  \begin{itemize}
  \item {\tt by} is a version of \verb=tapply= oriented towards data
    frames
  \item First argument is a data frame, others are as in \verb=tapply=
  \item \verb=by= returns a list
  \end{itemize}
\begin{knitrout}\tiny
\definecolor{shadecolor}{rgb}{0.969, 0.969, 0.969}\color{fgcolor}\begin{kframe}
\begin{alltt}
\hlcom{## Don't work, because iris is a data frame}
\hlkwd{by}\hlstd{(iris[,} \hlnum{1}\hlopt{:}\hlnum{4}\hlstd{], iris}\hlopt{$}\hlstd{Species, mean)}
\end{alltt}


{\ttfamily\noindent\color{warningcolor}{\#\# Warning in mean.default(data[x, , drop = FALSE], ...): argument is not numeric or logical: returning NA}}

{\ttfamily\noindent\color{warningcolor}{\#\# Warning in mean.default(data[x, , drop = FALSE], ...): argument is not numeric or logical: returning NA}}

{\ttfamily\noindent\color{warningcolor}{\#\# Warning in mean.default(data[x, , drop = FALSE], ...): argument is not numeric or logical: returning NA}}\begin{verbatim}
## iris$Species: setosa
## [1] NA
## -------------------------------------------------------- 
## iris$Species: versicolor
## [1] NA
## -------------------------------------------------------- 
## iris$Species: virginica
## [1] NA
\end{verbatim}
\end{kframe}
\end{knitrout}

\end{frame}

\begin{frame}[fragile]
  \ft{Groups Based on One or More Grouping Variables}
  \fst{{\tt by}}
  
\begin{knitrout}\scriptsize
\definecolor{shadecolor}{rgb}{0.969, 0.969, 0.969}\color{fgcolor}\begin{kframe}
\begin{alltt}
\hlstd{(ex_by} \hlkwb{<-} \hlkwd{by}\hlstd{(iris[,} \hlnum{1}\hlopt{:}\hlnum{4}\hlstd{], iris[,} \hlstr{"Species"}\hlstd{], colMeans))}
\end{alltt}
\begin{verbatim}
## iris[, "Species"]: setosa
## Sepal.Length  Sepal.Width Petal.Length  Petal.Width 
##        5.006        3.428        1.462        0.246 
## -------------------------------------------------------- 
## iris[, "Species"]: versicolor
## Sepal.Length  Sepal.Width Petal.Length  Petal.Width 
##        5.936        2.770        4.260        1.326 
## -------------------------------------------------------- 
## iris[, "Species"]: virginica
## Sepal.Length  Sepal.Width Petal.Length  Petal.Width 
##        6.588        2.974        5.552        2.026
\end{verbatim}
\end{kframe}
\end{knitrout}
  
\end{frame}

\begin{frame}[fragile]
  \ft{Groups Based on One or More Grouping Variables}
  \fst{{\tt by}}

\begin{knitrout}\scriptsize
\definecolor{shadecolor}{rgb}{0.969, 0.969, 0.969}\color{fgcolor}\begin{kframe}
\begin{alltt}
\hlkwd{do.call}\hlstd{(rbind, ex_by)}
\end{alltt}
\begin{verbatim}
##            Sepal.Length Sepal.Width Petal.Length Petal.Width
## setosa            5.006       3.428        1.462       0.246
## versicolor        5.936       2.770        4.260       1.326
## virginica         6.588       2.974        5.552       2.026
\end{verbatim}
\end{kframe}
\end{knitrout}
\begin{itemize}
\item \verb=do.call= takes a {\em list} of arguments (second argument)
\item and prepares a call to a function (first argument), using the
  list elements as if they had been passed to the function as
  individual arguments 
\end{itemize}
\end{frame}

\begin{frame}[fragile]
  \ft{Groups Based on One or More Grouping Variables}
  \fst{Split-Apply-Combine}
  
  Term coined by Hadley Wickham (author of the {\bf ggplot2}, {\bf plyr},
  {\bf reshape}, {\bf dplyr}, \dots, packages)
  \begin{description}
  \item[Split] Divide the problem into smaller pieces
  \item[Apply] Work on each pieces independently
  \item[Combine] Recombine the pieces
  \end{description}
  A common problem for both programming and data analysis; many
  implementations
  \begin{itemize}
  \item In base R: {\tt split()}, {\tt *apply()}, {\tt do.call()}
  \item R-packages: {\bf plyr}, {\bf doBy}, {\bf dplyr}, {\bf
      data.table} (to some extent) 
  \end{itemize}
\end{frame}

\begin{frame}[fragile]
  \ft{Split-Apply-Combine}
  \fst{Base R}
  \begin{itemize}
  \item Split by species
\begin{knitrout}\tiny
\definecolor{shadecolor}{rgb}{0.969, 0.969, 0.969}\color{fgcolor}\begin{kframe}
\begin{alltt}
\hlstd{s_iris} \hlkwb{<-} \hlkwd{split}\hlstd{(iris, iris}\hlopt{$}\hlstd{Species)}

\hlcom{## s_iris is a list with number of items }
\hlcom{## equal to the number of levels of iris$Species}
\hlkwd{length}\hlstd{(s_iris)} \hlopt{==} \hlkwd{length}\hlstd{(}\hlkwd{levels}\hlstd{(iris}\hlopt{$}\hlstd{Species))}
\end{alltt}
\begin{verbatim}
## [1] TRUE
\end{verbatim}
\end{kframe}
\end{knitrout}
\item Apply a function to each item of the list
\begin{knitrout}\tiny
\definecolor{shadecolor}{rgb}{0.969, 0.969, 0.969}\color{fgcolor}\begin{kframe}
\begin{alltt}
\hlstd{s_means} \hlkwb{<-} \hlkwd{lapply}\hlstd{(s_iris,} \hlkwa{function}\hlstd{(}\hlkwc{x}\hlstd{)} \hlkwd{colMeans}\hlstd{(x[}\hlnum{1}\hlopt{:}\hlnum{4}\hlstd{]))}
\hlstd{s_means[[}\hlnum{1}\hlstd{]]}
\end{alltt}
\begin{verbatim}
## Sepal.Length  Sepal.Width Petal.Length  Petal.Width 
##        5.006        3.428        1.462        0.246
\end{verbatim}
\end{kframe}
\end{knitrout}
\item Combine
\begin{knitrout}\tiny
\definecolor{shadecolor}{rgb}{0.969, 0.969, 0.969}\color{fgcolor}\begin{kframe}
\begin{alltt}
\hlstd{(res} \hlkwb{<-} \hlkwd{do.call}\hlstd{(rbind, s_means))}
\end{alltt}
\begin{verbatim}
##            Sepal.Length Sepal.Width Petal.Length Petal.Width
## setosa            5.006       3.428        1.462       0.246
## versicolor        5.936       2.770        4.260       1.326
## virginica         6.588       2.974        5.552       2.026
\end{verbatim}
\end{kframe}
\end{knitrout}

\end{itemize}
  
\end{frame}

\begin{frame}[fragile]
  \ft{Split-Apply-Combine}
  \fst{Base R}
  
\begin{knitrout}\tiny
\definecolor{shadecolor}{rgb}{0.969, 0.969, 0.969}\color{fgcolor}\begin{kframe}
\begin{alltt}
\hlstd{myLM} \hlkwb{<-} \hlkwa{function}\hlstd{(}\hlkwc{x}\hlstd{) \{}
    \hlstd{temp} \hlkwb{<-} \hlkwd{lm}\hlstd{(Sepal.Length} \hlopt{~} \hlstd{Petal.Length, x)}
    \hlkwd{summary}\hlstd{(temp)}\hlopt{$}\hlstd{coefficients}
\hlstd{\}}

\hlstd{(res} \hlkwb{<-} \hlkwd{lapply}\hlstd{(}\hlkwd{split}\hlstd{(iris, iris}\hlopt{$}\hlstd{Species), myLM))}
\end{alltt}
\begin{verbatim}
## $setosa
##               Estimate Std. Error   t value     Pr(>|t|)
## (Intercept)  4.2131682  0.4155888 10.137830 1.614927e-13
## Petal.Length 0.5422926  0.2823153  1.920876 6.069778e-02
## 
## $versicolor
##              Estimate Std. Error  t value     Pr(>|t|)
## (Intercept)  2.407523  0.4462583 5.394909 2.075294e-06
## Petal.Length 0.828281  0.1041364 7.953806 2.586190e-10
## 
## $virginica
##               Estimate Std. Error   t value     Pr(>|t|)
## (Intercept)  1.0596591 0.46676645  2.270213 2.772289e-02
## Petal.Length 0.9957386 0.08366764 11.901120 6.297786e-16
\end{verbatim}
\end{kframe}
\end{knitrout}

\end{frame}

%% \begin{frame}[fragile]
%%     \ft{Split-Apply-Combine}
%%     \fst{The {\bf plyr} Package}
    
%%     The *apply functions in base R implement the split-apply-combine
%%     strategy, but are inconsistent
%%     \begin{itemize}
%%     \item \verb=apply()= input {\tt arrays}; split by row and/or columns;
%%       output {\tt array} 
%%     \item \verb=lapply()= input {\tt list} or {\tt vector}; output
%%       {\tt list}
%%     \item \verb=sapply()= input {\tt list} or {\tt vector}; simplify
%%       to vector
%%     \item \verb=tapply= input {\tt data.frame}; output depends
%%     \item \verb=rapply()=, \verb=vapply()=, \verb=mapply()=
%%     \end{itemize}\pause\vspace{0.3cm}
%%     {\bf plyr} brings some consistency: \verb=**ply()=
%%     \begin{description}
%%     \item[first *] Input type ({\tt a} array, {\tt d} data frame, {\tt
%%         l} list)
%%     \item[second *] Output type ({\tt a} array, {\tt d} data frame, {\tt
%%         l} list, \verb=_= discard)
%%     \end{description}
  
%% \end{frame}

%% \begin{frame}[fragile]
%%   \ft{{\bf plyr}}
%%   \fst{{\tt a*ply()}}
  
%%   \verb=y <- a*ply(.data, .margins., .fun, ...)=
%%   \vspace{0.5cm}
%%   \begin{description}
%%   \item[.data] An array
%%   \item[.margins] Subscripts which the function gets applied over
%%   \item[.fun] Function to apply to each piece
%%   \end{description}
%%   Returns an array (\verb|*=a|), a data.frame (\verb|*=d|), a list
%%   (\verb|*=l|)
  
%% \end{frame}

%% \begin{frame}[fragile]
%%   \ft{{\bf plyr}}
%%   \fst{{\tt l*ply()}}
  
%%   \verb=y <- l*ply(.data, .fun, ...)=
%%   \vspace{0.5cm}
%%   \begin{description}
%%   \item[.data] An list
%%   \item[.fun] Function to apply to each item of the list
%%   \end{description}
%%   Returns an array (\verb|*=a|), a data.frame (\verb|*=d|), a list
%%   (\verb|*=l|)
  
%% \end{frame}

%% \begin{frame}[fragile]
%%   \ft{{\bf plyr}}
%%   \fst{{\tt d*ply()}}
  
%%   \verb=y <- d*ply(.data, .variables, .fun, ...)=
%%   \vspace{0.5cm}
%%   \begin{description}
%%   \item[.data] A data frame
%%   \item[.variables] Variables defining the groups
%%   \item[.fun] Function to apply to each group
%%   \end{description}
%%   Returns an array (\verb|*=a|), a data.frame (\verb|*=d|), a list
%%   (\verb|*=l|)
  
%% \end{frame}

%% \begin{frame}[fragile]
%%   \ft{{\bf plyr}}
%%   \fst{{\tt d*ply()}}

%% <<ddply1, size = "scriptsize">>=
%% (res <- ddply(iris, "Species", myLM))
%% @ %def 

%% \pause \vspace{0.5cm} 
%% The only problem with {\bf plyr} is that it
%% is sometimes slow
%% \end{frame}

\begin{frame}
  \ft{The {\bf dplyr} package} 
  
  The {\bf dplyr} package proposes a ``grammar of data manipulation'',
  i.e., it implements ``verbs'' useful for data manipulation.
  
  \begin{description}
  \item[{\tt select}] column subset (select variables)
  \item[{\tt filter}] row subset ($\Leftrightarrow$ {\tt subset} in
    base R)
  \item[mutate] add new/modify rows
  \item[summarise] summary statistics
  \item[{\tt arrange}] re-order the rows
  \item[{\tt do}] arbitrary action
  \end{description}
  
  \begin{itemize}
  \item {\bf dplyr} supports data.frames, data.tables (see later) as
    well as data bases
  \item Operations can be chained using a pipe operator
  \end{itemize}
  
\end{frame}

\begin{frame}[fragile]
  \ft{The {\bf dplyr} package}
  
  Compute the mean sepal width by species for flower whose petal
  length is longer than 4.8
\begin{knitrout}\scriptsize
\definecolor{shadecolor}{rgb}{0.969, 0.969, 0.969}\color{fgcolor}\begin{kframe}
\begin{alltt}
\hlkwd{require}\hlstd{(dplyr)}
\end{alltt}


{\ttfamily\noindent\itshape\color{messagecolor}{\#\# Loading required package: dplyr\\\#\# \\\#\# Attaching package: 'dplyr'\\\#\# \\\#\# The following object is masked from 'package:stats':\\\#\# \\\#\#\ \ \ \  filter\\\#\# \\\#\# The following objects are masked from 'package:base':\\\#\# \\\#\#\ \ \ \  intersect, setdiff, setequal, union}}\begin{alltt}
\hlstd{iris} \hlopt \hlkwd{group_by}\hlstd{(Species)} \hlopt \hlkwd{filter}\hlstd{(Petal.Length} \hlopt{>} \hlnum{4.8}\hlstd{)} \hlopt
  \hlkwd{summarise}\hlstd{(}\hlkwc{mean_width} \hlstd{=} \hlkwd{mean}\hlstd{(Petal.Width))}
\end{alltt}
\begin{verbatim}
## Source: local data frame [2 x 2]
## 
##      Species mean_width
## 1 versicolor   1.575000
## 2  virginica   2.042553
\end{verbatim}
\end{kframe}
\end{knitrout}
  
\end{frame}

\begin{frame}[fragile]
  \ft{The {\bf dplyr} package}
  
  Compute a linear model per species and petal length and summarise
  the results
  
\begin{knitrout}\tiny
\definecolor{shadecolor}{rgb}{0.969, 0.969, 0.969}\color{fgcolor}\begin{kframe}
\begin{alltt}
\hlstd{iris} \hlopt \hlkwd{mutate}\hlstd{(}\hlkwc{Petal.Length.f} \hlstd{=} \hlkwd{factor}\hlstd{(iris}\hlopt{$}\hlstd{Petal.Length} \hlopt{>} \hlnum{4.8}\hlstd{,}
                \hlkwc{levels} \hlstd{=} \hlkwd{c}\hlstd{(}\hlnum{FALSE}\hlstd{,} \hlnum{TRUE}\hlstd{),}
                \hlkwc{labels} \hlstd{=} \hlkwd{c}\hlstd{(}\hlstr{"Small petals"}\hlstd{,} \hlstr{"Big petals"}\hlstd{)))} \hlopt
  \hlkwd{group_by}\hlstd{(Species, Petal.Length.f)} \hlopt
  \hlkwd{do}\hlstd{(}\hlkwc{mod} \hlstd{=} \hlkwd{summary}\hlstd{(}\hlkwd{lm}\hlstd{(Sepal.Length} \hlopt{~} \hlstd{Petal.Length,} \hlkwc{data} \hlstd{= .))}\hlopt{$}\hlstd{coefficients)} \hlopt
  \hlkwd{do}\hlstd{(}\hlkwd{data.frame}\hlstd{(}
         \hlkwc{Species} \hlstd{= .}\hlopt{$}\hlstd{Species,}
         \hlkwc{Petal.Length.f} \hlstd{= .}\hlopt{$}\hlstd{Petal.Length.f,}
         \hlkwc{var} \hlstd{=} \hlkwd{rownames}\hlstd{(.}\hlopt{$}\hlstd{mod),}
         \hlkwc{coef} \hlstd{= .}\hlopt{$}\hlstd{mod[,} \hlnum{1}\hlstd{],}
         \hlkwc{se} \hlstd{= .}\hlopt{$}\hlstd{mod[,} \hlnum{2}\hlstd{],}
         \hlkwc{p} \hlstd{=} \hlkwd{format.pval}\hlstd{(.}\hlopt{$}\hlstd{mod[,} \hlnum{4}\hlstd{],} \hlkwc{eps} \hlstd{=} \hlnum{10}\hlopt{^}\hlstd{(}\hlopt{-}\hlnum{3}\hlstd{),} \hlkwc{digits} \hlstd{=} \hlnum{2}\hlstd{)))}
\end{alltt}


{\ttfamily\noindent\color{warningcolor}{\#\# Warning in rbind\_all(out[[1]]): Unequal factor levels: coercing to character}}\begin{verbatim}
## Source: local data frame [10 x 6]
## Groups: <by row>
## 
##       Species Petal.Length.f          var        coef          se      p
## 1      setosa   Small petals  (Intercept)   4.2131682  0.41558877 <0.001
## 2      setosa   Small petals Petal.Length   0.5422926  0.28231526  0.061
## 3  versicolor   Small petals  (Intercept)   2.2844372  0.48523067 <0.001
## 4  versicolor   Small petals Petal.Length   0.8587048  0.11498472 <0.001
## 5  versicolor     Big petals  (Intercept)  19.5909091 11.55006708   0.23
## 6  versicolor     Big petals Petal.Length  -2.6363636  2.32129915   0.37
## 7   virginica   Small petals  (Intercept) -13.1000000  2.71477439  0.130
## 8   virginica   Small petals Petal.Length   4.0000000  0.57735027  0.091
## 9   virginica     Big petals  (Intercept)   1.1908726  0.49934706  0.021
## 10  virginica     Big petals Petal.Length   0.9727855  0.08869172 <0.001
\end{verbatim}
\end{kframe}
\end{knitrout}

\end{frame}

\begin{frame}[fragile]
  \ft{The {\bf data.table} package}
  
  The {\bf data.table} package enhances the base data.frame. The
  package offers (extremely) fast
  
  \begin{itemize}
  \item subset
  \item grouping
  \item update
  \item joints (merging)
  \end{itemize}
  
  A {\tt data.table} inherits from {\tt data.frame}, i.e., it is
  compatible with R functions and packages that only accept
  {\tt data.frame}.
  
\end{frame}

\begin{frame}[fragile]
  \ft{The {\bf data.table} package}
  
  The general syntax is
\begin{knitrout}\scriptsize
\definecolor{shadecolor}{rgb}{0.969, 0.969, 0.969}\color{fgcolor}\begin{kframe}
\begin{alltt}
\hlstd{dt[i, j, by]}
\end{alltt}
\end{kframe}
\end{knitrout}
  \begin{description}
  \item[{\tt i}] permits to select rows (A bit like {\tt subset})
  \item[{\tt j}] permits to update/create columns. Extremely flexible
    {\color{gray}(maybe too much?)}
   \item[{\tt by}] permits to ``group by''
  \end{description}
Additionally, data.tables can be {\em keyed} by one or more variables,
leading to
\begin{itemize}
\item ordered data
\item faster merging by keyed variables
\end{itemize}

\end{frame}

\begin{frame}[fragile]
  \ft{The {\bf data.table} package}
  
  Subset rows in {\tt i}
\begin{knitrout}\scriptsize
\definecolor{shadecolor}{rgb}{0.969, 0.969, 0.969}\color{fgcolor}\begin{kframe}
\begin{alltt}
\hlkwd{require}\hlstd{(data.table)}
\end{alltt}


{\ttfamily\noindent\itshape\color{messagecolor}{\#\# Loading required package: data.table\\\#\# data.table 1.9.4\ \ For help type: ?data.table\\\#\# *** NB: by=.EACHI is now explicit. See README to restore previous behaviour.\\\#\# \\\#\# Attaching package: 'data.table'\\\#\# \\\#\# The following objects are masked from 'package:dplyr':\\\#\# \\\#\#\ \ \ \  between, last}}\begin{alltt}
\hlstd{(dt_iris} \hlkwb{<-} \hlkwd{data.table}\hlstd{(iris,} \hlkwc{key} \hlstd{=} \hlstr{"Species"}\hlstd{))}
\end{alltt}
\begin{verbatim}
##      Sepal.Length Sepal.Width Petal.Length Petal.Width   Species
##   1:          5.1         3.5          1.4         0.2    setosa
##   2:          4.9         3.0          1.4         0.2    setosa
##   3:          4.7         3.2          1.3         0.2    setosa
##   4:          4.6         3.1          1.5         0.2    setosa
##   5:          5.0         3.6          1.4         0.2    setosa
##  ---                                                            
## 146:          6.7         3.0          5.2         2.3 virginica
## 147:          6.3         2.5          5.0         1.9 virginica
## 148:          6.5         3.0          5.2         2.0 virginica
## 149:          6.2         3.4          5.4         2.3 virginica
## 150:          5.9         3.0          5.1         1.8 virginica
##      sepal_width sepal_widthNA Petal.Length.f
##   1:      <= 3.5            NA   Small petals
##   2:      <= 3.5        <= 3.5   Small petals
##   3:      <= 3.5        <= 3.5   Small petals
##   4:      <= 3.5        <= 3.5   Small petals
##   5:       > 3.5         > 3.5   Small petals
##  ---                                         
## 146:      <= 3.5        <= 3.5     Big petals
## 147:      <= 3.5        <= 3.5     Big petals
## 148:      <= 3.5        <= 3.5     Big petals
## 149:      <= 3.5        <= 3.5     Big petals
## 150:      <= 3.5        <= 3.5     Big petals
\end{verbatim}
\end{kframe}
\end{knitrout}

\end{frame}

\begin{frame}[fragile]
  \ft{The {\bf data.table} package}
  
  Subset rows in {\tt i}

\begin{knitrout}\scriptsize
\definecolor{shadecolor}{rgb}{0.969, 0.969, 0.969}\color{fgcolor}\begin{kframe}
\begin{alltt}
\hlstd{dt_iris[Species} \hlopt{==} \hlstr{"versicolor"} \hlopt{&} \hlstd{Petal.Length} \hlopt{>} \hlnum{4.8}\hlstd{]}
\end{alltt}
\begin{verbatim}
##    Sepal.Length Sepal.Width Petal.Length Petal.Width    Species
## 1:          6.9         3.1          4.9         1.5 versicolor
## 2:          6.3         2.5          4.9         1.5 versicolor
## 3:          6.7         3.0          5.0         1.7 versicolor
## 4:          6.0         2.7          5.1         1.6 versicolor
##    sepal_width sepal_widthNA Petal.Length.f
## 1:      <= 3.5        <= 3.5     Big petals
## 2:      <= 3.5        <= 3.5     Big petals
## 3:      <= 3.5        <= 3.5     Big petals
## 4:      <= 3.5        <= 3.5     Big petals
\end{verbatim}
\end{kframe}
\end{knitrout}

\end{frame}

\begin{frame}[fragile]
  \ft{The {\bf data.table} package}
  
  Select columns in {\tt j}
\begin{knitrout}\scriptsize
\definecolor{shadecolor}{rgb}{0.969, 0.969, 0.969}\color{fgcolor}\begin{kframe}
\begin{alltt}
\hlstd{dt_iris[}\hlnum{1}\hlopt{:}\hlnum{3}\hlstd{, Species]}
\end{alltt}
\begin{verbatim}
## [1] setosa setosa setosa
## Levels: setosa versicolor virginica
\end{verbatim}
\begin{alltt}
\hlstd{dt_iris[}\hlnum{1}\hlopt{:}\hlnum{3}\hlstd{,} \hlkwd{.}\hlstd{(Species, Petal.Length)]}
\end{alltt}
\begin{verbatim}
##    Species Petal.Length
## 1:  setosa          1.4
## 2:  setosa          1.4
## 3:  setosa          1.3
\end{verbatim}
\begin{alltt}
\hlstd{dt_iris[Species} \hlopt{==} \hlstr{"versicolor"} \hlopt{&} \hlstd{Petal.Length} \hlopt{>} \hlnum{4.8}\hlstd{,}
        \hlkwd{.}\hlstd{(Species, Petal.Length)]}
\end{alltt}
\begin{verbatim}
##       Species Petal.Length
## 1: versicolor          4.9
## 2: versicolor          4.9
## 3: versicolor          5.0
## 4: versicolor          5.1
\end{verbatim}
\end{kframe}
\end{knitrout}
\end{frame}

\begin{frame}[fragile]
  \ft{The {\bf data.table} package}
  
  Compute in {\tt j}: As long as {\tt j-expressions} returns a list,
  each element of the list will be converted to a column
  
\begin{knitrout}\scriptsize
\definecolor{shadecolor}{rgb}{0.969, 0.969, 0.969}\color{fgcolor}\begin{kframe}
\begin{alltt}
\hlstd{dt_iris[,} \hlkwd{mean}\hlstd{(Petal.Length)]}
\end{alltt}
\begin{verbatim}
## [1] 3.758
\end{verbatim}
\begin{alltt}
\hlstd{dt_iris[Species} \hlopt{==} \hlstr{"versicolor"}\hlstd{,} \hlkwd{mean}\hlstd{(Petal.Length)]}
\end{alltt}
\begin{verbatim}
## [1] 4.26
\end{verbatim}
\end{kframe}
\end{knitrout}
\end{frame}

\begin{frame}[fragile]
  \ft{The {\bf data.table} package}

\begin{knitrout}\scriptsize
\definecolor{shadecolor}{rgb}{0.969, 0.969, 0.969}\color{fgcolor}\begin{kframe}
\begin{alltt}
\hlstd{dt_iris[Species} \hlopt{==} \hlstr{"versicolor"}\hlstd{,} \hlkwd{.}\hlstd{(}\hlkwc{mean} \hlstd{=} \hlkwd{mean}\hlstd{(Petal.Length),}
        \hlkwc{sd} \hlstd{=} \hlkwd{sd}\hlstd{(Petal.Length))]}
\end{alltt}
\begin{verbatim}
##    mean       sd
## 1: 4.26 0.469911
\end{verbatim}
\begin{alltt}
\hlcom{## With a use defined function}
\hlstd{myFun} \hlkwb{<-} \hlkwa{function}\hlstd{(}\hlkwc{x}\hlstd{) \{}
    \hlkwd{list}\hlstd{(}\hlkwc{mean} \hlstd{=} \hlkwd{mean}\hlstd{(x),}
         \hlkwc{sd} \hlstd{=} \hlkwd{sd}\hlstd{(x))}
\hlstd{\}}
\hlstd{dt_iris[Species} \hlopt{==} \hlstr{"versicolor"}\hlstd{,} \hlkwd{myFun}\hlstd{(Petal.Length)]}
\end{alltt}
\begin{verbatim}
##    mean       sd
## 1: 4.26 0.469911
\end{verbatim}
\end{kframe}
\end{knitrout}

Special symbol {\tt .N}
\begin{knitrout}\scriptsize
\definecolor{shadecolor}{rgb}{0.969, 0.969, 0.969}\color{fgcolor}\begin{kframe}
\begin{alltt}
\hlstd{dt_iris[Species} \hlopt{==} \hlstr{"versicolor"} \hlopt{&} \hlstd{Petal.Length} \hlopt{>} \hlnum{4.8}\hlstd{,}
        \hlstd{.N]}
\end{alltt}
\begin{verbatim}
## [1] 4
\end{verbatim}
\end{kframe}
\end{knitrout}
\end{frame}

\begin{frame}[fragile]
  \ft{The {\bf data.table} package}
  
  Group by using {\tt by}
\begin{knitrout}\scriptsize
\definecolor{shadecolor}{rgb}{0.969, 0.969, 0.969}\color{fgcolor}\begin{kframe}
\begin{alltt}
\hlstd{dt_iris[, .N,} \hlkwc{by} \hlstd{=} \hlkwd{.}\hlstd{(Species, Petal.Length.f)]}
\end{alltt}
\begin{verbatim}
##       Species Petal.Length.f  N
## 1:     setosa   Small petals 50
## 2: versicolor   Small petals 46
## 3: versicolor     Big petals  4
## 4:  virginica     Big petals 47
## 5:  virginica   Small petals  3
\end{verbatim}
\begin{alltt}
\hlstd{dt_iris[,} \hlkwd{myFun}\hlstd{(Sepal.Length),} \hlkwc{by} \hlstd{=} \hlkwd{.}\hlstd{(Species, Petal.Length.f)]}
\end{alltt}
\begin{verbatim}
##       Species Petal.Length.f     mean        sd
## 1:     setosa   Small petals 5.006000 0.3524897
## 2: versicolor   Small petals 5.889130 0.5012111
## 3: versicolor     Big petals 6.475000 0.4031129
## 4:  virginica     Big petals 6.644681 0.5955664
## 5:  virginica   Small petals 5.700000 0.7000000
\end{verbatim}
\end{kframe}
\end{knitrout}

\end{frame}

\begin{frame}[fragile]
  \ft{The {\bf data.table} package}

  Reorder the last output by {\tt Species} and {\tt Petal.Length.f}

\begin{knitrout}\scriptsize
\definecolor{shadecolor}{rgb}{0.969, 0.969, 0.969}\color{fgcolor}\begin{kframe}
\begin{alltt}
\hlstd{tmp} \hlkwb{<-} \hlstd{dt_iris[,} \hlkwd{myFun}\hlstd{(Sepal.Length),} \hlkwc{by} \hlstd{=} \hlkwd{.}\hlstd{(Species, Petal.Length.f)]}
\hlstd{(}\hlkwd{setkeyv}\hlstd{(tmp,} \hlkwd{c}\hlstd{(}\hlstr{"Species"}\hlstd{,} \hlstr{"Petal.Length.f"}\hlstd{)))}
\end{alltt}
\begin{verbatim}
##       Species Petal.Length.f     mean        sd
## 1:     setosa   Small petals 5.006000 0.3524897
## 2: versicolor   Small petals 5.889130 0.5012111
## 3: versicolor     Big petals 6.475000 0.4031129
## 4:  virginica   Small petals 5.700000 0.7000000
## 5:  virginica     Big petals 6.644681 0.5955664
\end{verbatim}
\end{kframe}
\end{knitrout}
\end{frame}

\begin{frame}[fragile]
  \ft{The {\bf data.table} package}
  
  Chaining:
\begin{knitrout}\tiny
\definecolor{shadecolor}{rgb}{0.969, 0.969, 0.969}\color{fgcolor}\begin{kframe}
\begin{alltt}
\hlstd{my_lm} \hlkwb{<-} \hlkwa{function}\hlstd{(}\hlkwc{y}\hlstd{,} \hlkwc{x}\hlstd{) \{}
    \hlstd{mod} \hlkwb{<-} \hlkwd{summary}\hlstd{(}\hlkwd{lm}\hlstd{(y} \hlopt{~} \hlstd{x))}\hlopt{$}\hlstd{coefficients}
    \hlkwd{list}\hlstd{(}\hlkwc{var} \hlstd{=} \hlkwd{rownames}\hlstd{(mod),}
         \hlkwc{coef} \hlstd{= mod[,} \hlnum{1}\hlstd{],}
         \hlkwc{se} \hlstd{= mod[,} \hlnum{2}\hlstd{],}
         \hlkwc{p} \hlstd{=} \hlkwd{format.pval}\hlstd{(mod[,} \hlnum{4}\hlstd{],} \hlkwc{eps} \hlstd{=} \hlnum{10}\hlopt{^}\hlstd{(}\hlopt{-}\hlnum{3}\hlstd{),} \hlkwc{digits} \hlstd{=} \hlnum{2}\hlstd{))}
\hlstd{\}}

\hlstd{dt_iris[, Petal.Length.f} \hlkwb{:=} \hlkwd{factor}\hlstd{(Petal.Length} \hlopt{>} \hlnum{4.8}\hlstd{,}
          \hlkwc{levels} \hlstd{=} \hlkwd{c}\hlstd{(}\hlnum{FALSE}\hlstd{,} \hlnum{TRUE}\hlstd{),}
          \hlkwc{labels} \hlstd{=} \hlkwd{c}\hlstd{(}\hlstr{"Small petals"}\hlstd{,} \hlstr{"Big petals"}\hlstd{))}
        \hlstd{][,} \hlkwd{my_lm}\hlstd{(Sepal.Length, Petal.Length),}
          \hlkwc{by} \hlstd{=} \hlkwd{.}\hlstd{(Species, Petal.Length.f)}
          \hlstd{][}\hlkwd{order}\hlstd{(Species, Petal.Length.f)]}
\end{alltt}
\begin{verbatim}
##        Species Petal.Length.f         var        coef          se      p
##  1:     setosa   Small petals (Intercept)   4.2131682  0.41558877 <0.001
##  2:     setosa   Small petals           x   0.5422926  0.28231526  0.061
##  3: versicolor   Small petals (Intercept)   2.2844372  0.48523067 <0.001
##  4: versicolor   Small petals           x   0.8587048  0.11498472 <0.001
##  5: versicolor     Big petals (Intercept)  19.5909091 11.55006708   0.23
##  6: versicolor     Big petals           x  -2.6363636  2.32129915   0.37
##  7:  virginica   Small petals (Intercept) -13.1000000  2.71477439  0.130
##  8:  virginica   Small petals           x   4.0000000  0.57735027  0.091
##  9:  virginica     Big petals (Intercept)   1.1908726  0.49934706  0.021
## 10:  virginica     Big petals           x   0.9727855  0.08869172 <0.001
\end{verbatim}
\end{kframe}
\end{knitrout}
\end{frame}


\begin{frame}[fragile]
  \ft{Your Turn}
  
  Consider \verb=WM_teams_2014.csv= data set that contains information
  on each player of the World Cup 2014, e.g., age, club, country, caps
  (number of plays for the national team). Using {\bf dplyr} or {\bf
    data.table}, 
  \begin{enumerate}
  \item Find the three oldest and youngest players for each country
  \item Create a data set with the mean (with 95\% confidence
    interval), median, 25\% and 75\% percentile of the players' age
    stratified on country and position
  \item Create a data set containing the clubs, number of players in
    each club that participate in the world cup. The data set should
    be ordered from highest to lowest
  \end{enumerate}
\end{frame}

\end{document}
