\documentclass[11pt, german]{article}

\usepackage{natbib}

%Webkram
\usepackage{url}
\usepackage{hyperref}

%AMSTeX
\usepackage{amsmath}
\usepackage{amssymb}
\usepackage{amsthm}
\usepackage{latexsym}
\usepackage{verbatim}
\usepackage{epsfig}
\usepackage{enumerate}
\usepackage{rotate}
%\usepackage[ddmmyyyy]{datetime}

\def\bea{\begin{eqnarray*}}
\def\eea{\end{eqnarray*}}
%Dense format
\setlength{\parindent}{0em} \setlength{\textwidth}{16cm} \setlength{\textheight}{25cm}
\setlength{\topmargin}{-0.5cm} \setlength{\oddsidemargin}{0cm} \setlength{\headheight}{0cm}
\setlength{\headsep}{0cm}
\newcommand{\nat}{{\it I\hspace{-0.12cm}N}}
\pagestyle{empty}


\newcommand{\p}{{\rm P}}
\newcommand{\erw}{{\rm E}}
\newcommand{\var}{{\rm var}}
\newcommand{\eins}{{\bf 1}}%{\mathbf{1}}
\newcommand{\dif}{{\rm d}}
\newcommand{\cif}{{\rm CIF}}
\newcommand{\Prob}{\mathbb{P}}
\newcommand{\R}{\mathbb{R}}
\newcommand{\D}{\mathrm{d}}

%Special commands:
\newcommand{\Bin}{\operatorname{Bin}} % Binomial Distribution
\newcommand{\NegBin}{\operatorname{NegBin}} % Negative Bin
\newcommand{\HypGeom}{\operatorname{HypGeom}} % Hypergeometric Distribution
\newcommand{\Pois}{\operatorname{Po}} % Hypergeometric Distribution
\newcommand{\Po}{\operatorname{Po}} %
\newcommand{\Exp}{\operatorname{Exp}} %
\newcommand{\Par}{\operatorname{Par}} %
\newcommand{\Ga}{\mathcal{G}a} %
\newcommand{\Be}{\mathcal{B}e} %
\newcommand{\Var}{\operatorname{Var}} %
\newcommand{\E}{\operatorname{E}} %
\newcommand{\Cov}{\operatorname{Cov}} %
\newcommand{\MSE}{\operatorname{MSE}}

\DeclareMathOperator{\Nor}{N} % Normal -
\DeclareMathOperator{\Log}{Log} % Logistische Verteilung -
\newcommand{\ml}[2][1]{% % für Maximum-Likelihood-Schätzer von #1
\ifthenelse{#1 = 1}%
 {\hat{#2}_{\scriptscriptstyle{ML}}}%
 {\hat{#2}^{#1}_{\scriptscriptstyle{ML}}}% z.B. für sigmadach^2
}

%%%%%%%%%%%%%%%%%%%%%%%%%%%%%%%%%%%%%%%%%%%%%%%%%%%%%%%%%%%%%%%%%%%%%%
% Environment for Aufgaben
%%%%%%%%%%%%%%%%%%%%%%%%%%%%%%%%%%%%%%%%%%%%%%%%%%%%%%%%%%%%%%%%%%%%%%

\newcounter{ka}


\newtheorem{exercise}{Problem}
\newenvironment{aufgabe}{\begin{exercise}\rm}{\end{exercise} \bigskip}

\newcommand{\footer}{ \vfill
  \mbox{}\hrulefill\\
  Return your solutions until 30.4.2015, 10h00 s.t. per email at
  \url{arthur.allignol@uni-ulm.de} You may turn in solutions in pairs
  (two students, one solution; but not more than two). You may answer
  in German or English}
%\xxivtime   Uhrzeit für letzte Änderung

\pagestyle{empty}

\renewcommand{\labelenumi}{(\alph{enumi})}


%===============
\begin{document}
%===============

\renewcommand{\baselinestretch}{1}

\hrulefill\\
{\bf Introduction to biostatistical computing} \hspace{\fill} Summer term 2015\\
Arthur Allignol \quad\quad Karin Schreiber \hspace{\fill} 23.4.2015\\[-1.2ex]
\mbox{}\hrulefill\\
\newline \renewcommand{\baselinestretch}{1}
\setcounter{ka}{0} \vspace{-0.5cm}



\begin{center}
\large{{\bf Exercise Sheet 1}}
\end{center}

\begin{aufgabe}{{\bf Televisions, physicians, and life expectancy}}

  The aim of this exercise is to produce your first reproducible
  report using {\bf Sweave} or {\bf knitr}. The scientific question to
  be answered is whether there is a relation between life expectancy
  and the number of people per television. The report should at least
  contain the following
  \begin{enumerate}
  \item A short description of the data and the aim of the analysis
  \item A graphical exploration of the relation between the number of
    people per television and life expectancy
  \item A regression model to answer whether the number of people per
    television set in a country is a useful predictor of that
    country's life expectancy
  \end{enumerate}
  The report should contain at least one graphic and possibly one
  table (generated with R) and be 1 to 2 pages long. The exercise will
  be considered a great success if we can compile the document without
  errors. You may use the file {\tt explore\_television.R} as
  inspiration as well as {\tt tele.Rmd} and/or {\tt tele.Rnw}. Please
  return both the .Rnw/.Rmd file and the resulting pdf or html
  document.
\end{aufgabe}

\footer
\end{document}
